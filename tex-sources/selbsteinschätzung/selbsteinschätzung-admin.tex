\documentclass[a4paper]{article}
\usepackage[english,german]{babel}
\usepackage[utf8]{inputenc}
\usepackage[T1]{fontenc}
\usepackage{ae}
\usepackage{comment}
\usepackage{enumerate}
\author{Christoph Kepler} 
\date{}
\title{
\normalfont
\normalsize
\huge{Projektselbsteinschätzung}
}
\begin{document}
\maketitle
\tableofcontents

\newpage

\section{Vorwort}

\paragraph{Dies ist die persönliche Einschätzung von Christoph Kepler (im weitere Text als "ich") zum Softwaretechnologieprojekt 2014 der Gruppe 17. Dieser Bericht ist die subjektive Sicht von mir auf dieses Projekt und repräsentiert nicht gezwungener Weise die Sicht der gesamten Gruppe auf das Projekt.}

\section{Lob}

\paragraph{Als besonders hervorhebenswert finde ich die Zusammenarbeit in der Gruppe. Da unsere Gruppenmitglieder aus verschiedenen Semestern und sogar Studiengängen kamen, war es anfangs etwas schwierig, Termine zu finden, welche für alle Mitglieder praktikabel waren. Es wurde aber ziemlich schnell eine Möglichkeit gefunden und der reibungslose Ablauf war gegeben.\\
Meine Motivation war zu Anfang des Projektes nicht gerade die Beste, da ich mich durch die Berichte von höheren Semestern beeinflussen lassen habe. Trotzdem konnte ich mich durch die gestellte Aufgabe motivieren und hatte eine guten Lerneffekt durch die Arbeit an der Anwendung erzielen. Außerdem war es die erste Arbeit die ich nach dem Prinzip der klassischen Projektplanung realisiert habe. Dies ist vor allem für das spätere Berufsleben ein großer Bonus.\\
Die angebotene Hilfe durch das Auditorium ist ein sehr großer Pluspunkt, bei dieser Arbeit, da eine Lösungsfindung vor allem bezüglich SALESPOINT sich nicht gerade einfach gestaltete. Außerdem ist vor allem die Hilfe von Seiten unseres Tutors hervorzuheben. Soweit es ihm möglich war, stand er uns mit Rat und Tat bei Seite. Auch die Einführungsveranstaltungen waren zumindest teilweise durchaus hilfreich. Die lockere Herangehensweise von Oliver Gierke machte das Zuhören sehr einfach.\\
Die freie Wahl des Versionsverwaltungssystem und die Erlaubnis zur Benutzung von GITHUB sind vor allem für mich selbst ein großer Pluspunkt. Durch den vorhandenen Issuetracker, die Integration von TRAVIS CI und die Möglichkeit der Benutzung eines Wikis, wurden viele Sachen erleichtert.\\
Das Ziel eine Webanwendung zu schaffen, erweist sich zumindest dann als Vorteil, als das man seine Resultate besser erkennen kann und die Einarbeitung in eine GUI Erstellung gering bleibt.}

\section{Kritik}

\paragraph{Ein großer Kritikpunkt meinerseits ist die Verwendung zweier vollkommen unbekannter Frameworks. Die Einarbeitung sowohl in SPRING als auch in SALESPOINT hat einen großen Teil der Entwurfs- als auch der Implementationsphase gebraucht. Die Einarbeitung in SPRING ist durchaus einfach, da die bereitgestellte Dokumentation sehr erschöpfend ist und man durch Suchanfragen durchaus Hilfe in Foren findet. Dies ist bei SALESPOINT leider nicht gegeben. Die bereitgestellte Dokumentation ist, soweit ich sie finden konnte, dürftig bis nicht vorhanden. Die einzige Anlaufstelle ist das Auditorium, wo durchaus Lösungen für die Probleme gefunden werden konnten. Das Problem ist bloß, das eine gewisse Antwortzeit mit einzuberechnen ist. Für diesen Zeitraum liegt das Projekt in den meisten Fällen brach.\\
Als Anwendungsbeispiel für die gelernten Designpatterns ist dieses Projekt aber absolut unpassend. Es ist nahezu unmöglich sinnvoll irgendwelche Patterns zu inkludieren. Da mag einerseits an der Verwendung des MVC-Patterns für das Projekt liegen, als auch an der Singleton-Natur der Beans von Spring. Somit verkommt diese Arbeit zu einer recht einseitigen Programmiererfahrung, da sich viele Funktionen in den verschiedenen Klassen doppeln.\\
Das anfänglich angebotene FUSIONFORGE der TU Dresden ist als SCM gerade so benutztbar. Das beginnt schon bei der fehlenden Weiterleitung auf "https" über das Fehlen eines guten Interface zur Codeinspektion bis hin zu den regelmäßigen Aussetzern der Mailingliste. Eine Erwähnung von Github Education während der Einführungsveranstaltungen, wäre durchaus vorteilhaft.\\
Letztendlich finde ich die Verwendung von Java zur Realisierung eines dynamischen Webprojekt unvorteilhaft. Schon alleine eine Skriptsprache würde vieles vereinfachen. Da ich neben der Uni schon arbeite und dort vor allem mit der Betreuung von Webprojekten betraut bin, bin ich der Meinung ein objektives Urteil darüber Fällen zu können. Die Verwendung einer dedizierten Websprache z.B. PHP ist für eine Webanwedung durchaus besser. Zwar sind einige Dinge vereinfacht worden durch die Benutzung von Thymeleaf. Trotzdem bleibt die Natur des Projektes zu starr.\\
Die zeitaufwändige Natur dieser Projektarbeit resultiert ziemlich schnell in einen hohen Stresspegel bei allen Gruppenmitgliedern, Die angesetzten 180h für diese Projektarbeit werden ziemlich schnell überschritten. Vor allem die Entwurfsphase ist ihrer Gestaltung ungünstig. Aufgrund mangelnder Erfahrung mit den vorgegebenen Frameworks als auch in Projektarbeiten der meisten, ist die Beleuchtung der Anwendung in Form von mehreren UML-Diagrammen durchaus hinderlich. Es sollte eher ein Einblick oder das Prinzip der Entwurfsphase vermittelt werden, wodurch eine Möglichkeit entsteht die Implementationsphase auszubauen, Gleiches gilt für die Erstellung der verschiedenen Diagramme, Hefte und Dokumentationen, welche sich meistens in ihren Inhalten doppeln und dadurch nur unnötig Zeit in Anspruch nehmen. Diese Zeit könnte sinnvoller für die Einarbeitung oder das Ausprobieren der Frameworks genutzt werden. Dadurch würde am Ende des Projektes ein wesentlich besseres Ergebnis bei allen Gruppen zu präsentieren sein.}

\section{Zusammenfassung}

\paragraph{Alles in allem ist diese Projektarbeit eine durchaus interessante Erfahrung. An manchen Stellen gibt es durchaus Verbesserungsbedarf. Trotz alledem war es uns möglich, das Projekt zu einem vorzeigbaren Abschluss zu bringen.}

\end{document}