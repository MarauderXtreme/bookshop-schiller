\documentclass[a4paper]{article}
 
\usepackage[english,german]{babel}
\usepackage[utf8]{inputenc}
\usepackage[T1]{fontenc}
\usepackage{ae}
\usepackage{comment}
\usepackage[bookmarks,bookmarksnumbered]{hyperref}
\usepackage{enumerate}
\usepackage{datetime}
\usepackage[table,xcdraw]{xcolor}
\usepackage{graphicx}
\usepackage{longtable}
\usepackage{hyperref}

\DeclareGraphicsExtensions{.pdf,.png,.jpg,.jpeg,.gif}
\newcommand{\horrule}[1]{\rule{\linewidth}{#1}} % Create horizontal rule command with 1 argument of height

\author{Gruppe 17}

\date{}

\title{
	\normalfont
	\normalsize 
	\huge{Entwicklerdokumentation Buchhandlung Schiller}
	\horrule{0.5pt}
	\paragraph{Version: 1.0}
	\paragraph{Status: Fertig}
	\paragraph{Stand: 19. November 2014}
	\horrule{2pt}
}

\begin{document}

\maketitle

\newpage
 
\section*{Historie}

\begin{comment}

|l|l|l|l|l|

\end{comment}

\begin{tabular}{|r|r|c|l|c|}
	\hline
	\rowcolor[HTML]{C0C0C0} 
	Version & Status    & Datum      & Bearbeiter       & Erläuterung    	\\ \hline
	0.1     & In Arbeit & 04.01.2015 & Christoph Kepler & Initial Commit 	\\ \hline
	1.0     & Fertig 	& 14.01.2015 & Christoph Kepler & Fertigstellung 	\\ \hline
	1.1     & Fertig 	& 15.01.2015 & Christoph Kepler & Ausformulierungen 	\\ \hline
	1.2     & Fertig 	& 15.01.2015 & Christoph Kepler & Ausformulierungen 	\\ \hline
\end{tabular}

\newpage

\tableofcontents

\newpage

\section{Einführung und Ziele}

\subsection{Aufgabenstellung}

\paragraph{Die Buchhandlung SCHILLER benötigt eine Verkaufsanwendung. Hauptsächlich ist eine Verkaufsanwendung für die Bücher zu implementieren. Jedoch hat der Geschäftsführer noch einige eigene Ideen. 
Die Anwendung benötigt, neben einer Artikelverwaltung auch eine Benutzerverwaltung. Zu jedem Buch muss mindestens der Autor, Verlag, die ISBN und eine kurze Inhaltsbeschreibung gespeichert werden. Eine Abbildung des Buchbundes anzuzeigen, würde die Attraktivität des Verkaufsprogrammes deutlich steigern. Die Bücher der Buchhandlung SCHILLER sind nach Genre in die Kategorien Fiktion, Sachbuch, Unterhaltung, Ratgeber unterteilt. Eine Möglichkeit zu Erweiterung und nachträglichem Hinzufügen weiterer Genres ist wünschenswert. Der Geschäftsinhaber denkt auch über ein Angebot von CDs und DVDs nach. Die Benutzerverwaltung soll einige wichtige Angaben zum Kunden liefern (Name, Kundennummer, Lieferadresse, etc.). 
Als zusätzliches Feature wünscht der Buchhandel SCHILLER sich einen Kalender auf der Homepage, welcher die wöchentlichen Lesungen aufführt, die in den Räumen der Buchhandlung stattfinden. Die Bezahlung der gekauften Bücher erfolgt über Rechnungsversand.}

\subsection{Qualitätsziele}

\paragraph{Die Entwickler der Gruppe 17 arbeiten auf eine vollkommen funktionsfähige, intuitiv bedienbare und den heutigen Webstandards angepasste Anwendung hin. Diese wird den höchsten Qualitätsstandards entsprechen. Um die Sicherheit der Applikation zu gewährleisten werden diverse Tests und Sicherheitsparameter implementiert.}

\subsection{Stakeholder}

\begin{longtable}{|l|p{125px}|p{125px}|}
	\hline
	\rowcolor[HTML]{C0C0C0}
	Rolle	& Beschreibung	& Ziel/Intention	\\ \hline
	Admin	& Administrativer Benutzer der Anwendung	& Administrieren der Anwendung	\\ \hline
	Boss	& Account des CEOs	& Einsehen von Statistiken	\\ \hline
	Article Manager	& Account des Article Mangers	& Verwaltung der Artikel	\\ \hline
	Event Manager	& Account des Event Mangers	& Verwaltung der Events	\\ \hline
	User Manager	& Account des User Mangers	& Verwaltung von Nutzern	\\ \hline
	Employee	& Accounts der Angestellten	& generelle Aufgaben	\\ \hline
	Customer	& Accounts der Kunden	& Für bestellen	\\ \hline
	Guest	& anonymer Websitebesucher	& surfen	\\ \hline
	
\end{longtable}

\newpage

\section{Randbedingungen}

\subsection{Technische Randbedingungen}

\begin{longtable}{|l|l|}
	\hline
	\rowcolor[HTML]{C0C0C0}
	\multicolumn{2}{|l|}{Hardwarevorgaben}	\\ \hline
	& min. 1024 MB RAM	\\ \hline
	& min. 2 GHz Single Core CPU	\\ \hline
	& min. 50 GB HDD Kapazität	\\ \hline
	& Breitband Internet Anbindung	\\ \hline
	\rowcolor[HTML]{C0C0C0}
	\multicolumn{2}{|l|}{Softwarevorgaben}	\\ \hline
	& Java 8 kompatibles OS	\\ \hline
	& Java 8	\\ \hline
	& Tomcat 8.0 oder höher	\\ \hline
	& Maven 3.0 oder höher	\\ \hline
	& konfigurierter Webserver/Proxy	\\ \hline
	& konfigurierte Domain	\\ \hline
	\rowcolor[HTML]{C0C0C0}
	\multicolumn{2}{|l|}{Vorgaben des Systembetriebs}	\\ \hline
	& CSS und JS Framework ZURB Foundation 5	\\ \hline
	& Foundation Date Picker	\\ \hline
	& Slick Image Slider	\\ \hline
	& Spring Framework 4	\\ \hline
	& Salespoint Framework 6	\\ \hline
	& itext PDF	\\ \hline
	& CK Editor	\\ \hline
	& Thymeleaf	\\ \hline
	\rowcolor[HTML]{C0C0C0}
	\multicolumn{2}{|l|}{Programmiervorgaben}	\\ \hline
	& Java 8	\\ \hline
	& HTML 5	\\ \hline
	& CSS 3	\\ \hline
	& JavaScript	\\ \hline
\end{longtable}

\subsection{Konventionen}

\textbf{Programmierrichtlinien}
\begin{itemize}
	\item 2 space tab intendation
	\item MVC Pattern
\end{itemize}
\textbf{Dokumentationsrichtlinien}
\begin{itemize}
	\item JavaDoc
	\item Wiki des Repository
	\item Kommentare im Quellcode
	\item Anwenderdoku
	\item Entwicklerdoku
\end{itemize}
\textbf{Richtlinien für das Versions- und Konfigurationsmangement}
\begin{itemize}
	\item Versionsmangement: Git (Github)
	\item Continous Integration: Tavis CI (Direktanbindung zu Github)
	\item Konfigurationsmangement: Maven mit pom.xml
\end{itemize}
\textbf{Namenskonventionen}
\begin{itemize}
	\item Namensraum bookshop-schiller
\end{itemize}

\section{Kontextabgrenzung}

\subsection{Fachlicher Kontext}

\subsubsection{Kontextdiagramm}

\includegraphics[width=350px]{kontextmodell.jpg}

\subsubsection{Top-Level-Architektur}

\includegraphics[width=350px]{top-level-architektur.jpg}

\subsection{Verteilungskontext}

\paragraph{Es wird nicht auf andere Systeme zugegriffen.}

\subsection{Externe Schnittstellen}

\subsubsection{Anwendungsfälle}

\includegraphics[width=350px]{use-case-diagramm-part1.jpg}

\includegraphics[width=350px]{use-case-diagramm-part2.jpg}

\includegraphics[width=350px]{use-case-diagramm-part3.jpg}

\begin{longtable}{|p{100px}|p{250px}|}
	\hline
	\rowcolor[HTML]{C0C0C0}
	Use-Case ID & Beschreibung \\ \hline
	\multicolumn{2}{|l|}{UserManagement:}  \\ \hline
	UC01 & LogIn/LogOut  \\ \hline
	UC02 & als Nutzer registrieren  \\ \hline
	UC03 & Rollen ändern \\ \hline
	UC04 & Nutzerdaten ändern  \\ \hline
	UC05 & eigene Nutzerdaten ansehen  \\ \hline
	UC06 & Nutzerdaten eines Kunden ansehen \\ \hline
	UC07 & Passwort zurücksetzen  \\ \hline
	UC08 & Nutzerprofil löschen \\ \hline
	UC09 & Nutzerprofil sperren \\ \hline
	\multicolumn{2}{|l|}{ArticleManagement:}  \\ \hline
	UC10 & Artikel ansehen (Buch, CD, DVD)  \\ \hline
	UC11 & Artikel hinzufügen  \\ \hline
	UC12 & Artikel entfernen  \\ \hline
	UC13 & Artikel nach verschiedenen Kriterien durchsuchen  \\ \hline
	UC14 & Artikeldaten ändern  \\ \hline
	UC15 & Kategorie bearbeiten  \\ \hline
	UC16 & Kategorie hinzufügen  \\ \hline
	UC17 & Kategorie entfernen  \\ \hline
	UC18 & Artikel zu Kategorie hinzufügen \\ \hline
	UC19 & Artikel aus Kategorie entfernen \\ \hline
	\multicolumn{2}{|l|}{CartManagement (SaleManagement):}  \\ \hline
	UC20 & Warenkorb ansehen  \\ \hline
	UC21 & Warenkorb füllen \\ \hline
	UC22 & Warenkorb leeren \\ \hline
	UC23 & zur Kasse gehen  \\ \hline
	\multicolumn{2}{|l|}{OrderManagement (SaleManagement):}  \\ \hline
	UC24 & Bestellung ansehen  \\ \hline
	UC25 & Bestellung abbrechen  \\ \hline
	UC26 & Retourenschein ausgeben  \\ \hline
	\multicolumn{2}{|l|}{BillManagement (SaleManagement):}  \\ \hline
	UC27 & Rechnung ansehen \\ \hline
	UC28 & Statistik ansehen \\ \hline
	\multicolumn{2}{|l|}{ReadingManagement:} \\ \hline
	UC29 & Kalender ansehen \\ \hline
	UC30 & Lesung hinzufügen \\ \hline
	UC31 & Lesung bearbeiten \\ \hline
	UC32 & Lesung entfernen \\ \hline
	UC33 & Raum hinzufügen \\ \hline
	UC34 & Raum bearbeiten \\ \hline
	UC35 & Raum entfernen \\ \hline
\end{longtable}

\section{Lösungsstrategien und Entwurfsenscheidungen}

\paragraph{Um die Aufgabenstellung zu realisieren wird das Java Framework Spring verwendet. Um die Verkaufsaspekte abzudecken, wird zusätzlich das Java Framework Salespoint eingesetzt. Diese beiden Frameworks sollen durch Wiederverwendung den Programmieraufwand so gering wie möglich halten. Die im SWT-Modul gelernten Design-Patterns sollen genutzt werden, um größtmögliche Modularisierung, einfache Erweiterbarkeit und unkomplizierte Wartung zu gewährleisten.}

\paragraph{Die zu erstellende Web Applikation stellt eine Verkaufsanwendung für die Buchandlung SCHILLER dar. Über diese Anwendung sollen Artikel in verschiedenen Kategorien zum Verkauf angeboten werden. Dazu müssen sich die Gäste an dem System mit ihrer korrekten E-Mail-Adresse registrieren. Es werden Rollen auf die einzelnen Nutzer verteilt, welche dadurch spezielle Berechtigungen auf das System erben. Für die verschiedenen Rollen ist es dann möglich ihrer Tätigkeit nach zu gehen. So kann zum Beispiel der Reading Manager die Lesungen in den dafür bereitgestellten Kalender und die Ressource Raum eintragen, solange nicht schon eine Lesung dort vorhanden ist. Der Rechnungsversand erfolgt automatisch als pdf via E-Mail an den Kunden.}

\section{Bausteinsicht}

\subsubsection{Kontextdiagramm}

\includegraphics[width=350px]{kontextmodell.jpg}

\subsubsection{Top-Level-Architektur}

\includegraphics[width=350px]{top-level-architektur.jpg}

\section{Laufzeitsicht}

\subsection{ArticleManagement}

\includegraphics[width=350px]{sd-articlemanagement.jpg}

\subsection{ProfileManagement}

\includegraphics[width=350px]{sd-profilemanagement.jpg}

\subsection{Purchase}

\includegraphics[width=350px]{sd-purchase.jpg}

\subsection{Registration}

\includegraphics[width=350px]{sd-registration.jpg}

\section{Konzepte}

\subsection{Fachliche Strukturen und Modelle}

\subsubsection{Überblick: Klassendiagramm}

\includegraphics[width=350px]{analyse-klassendiagramm.jpg}

\subsubsection{Klassen und Enumerationen}

\begin{longtable}{|p{100px}|p{250px}|}
	\hline
	\rowcolor[HTML]{C0C0C0} 
	Klasse/Enumeration & Beschreibung \\ \hline
	BookShopSchiller & Hauptklasse, die alle Funktionen zusammenführt. \\ \hline
	Guest & Klasse für einen  uneingeloggten Benutzer, kann den Einkaufswagen nutzen, sich über das Benutzermanagement einloggen oder kann sich registrieren, kann außerdem die Suchfunktion nutzen. \\ \hline
	Costumer & Klasse für einen eingeloggten Benutzer als Kunde, erbt von dem Guest und kann des weiteren auf sein Konto zugreifen. \\ \hline
	Employee & Klasse für einen Angestellten, erbt ebenfalls von Guest.  \\ \hline
	UserManager & Klasse erbt von Employee und kann außerdem auf die Verwaltung der Profile zugreifen. \\ \hline
	ReadingManager & Klasse erbt von Employee und kann außerdem auf die Verwaltung der Veranstaltungen zugreifen \\ \hline
	SaleManager & Klasse erbt von Employee und kann außerdem auf die Verwaltung der Bestellungen zugreifen. \\ \hline
	ArticleManager & Klasse erbt von Employee und kann außerdem auf die Verwaltung der Artikel zugreifen. \\ \hline
	Admin & Klasse erbt von den „Manager“-Klassen, um Gewalt über alle Funktionalitäten zu haben. \\ \hline
	Chef & Klasse erbt von Employee und kann außerdem Einsicht in die finanziellen Bilanzen bekommen. \\ \hline
	Lieferadresse & Enumeration gehört zu jedem Costumer. Sie enthält die Daten eines Kunden(Hausnummer, Straße, Etage) \\ \hline
	SaleManagement & Ist die Oberklasse zur Verwaltung/Funktionalität des Verkaufs. Sie regelt daher den Verkauf an Kunden. Der SaleManager benutzt die Funktionalitäten dieser Klasse. \\ \hline
	CartManagement & Klasse kümmert sich um die Verwaltung des Einkaufswagen durch einen Costumer. \\ \hline
	OrderManagement & Klasse verwaltet und bietet die Funktionalitäten einer jeden Bestellung. \\ \hline
	BalanceSheet & Klasse bietet Funktionalitäten zum Einsehen der Ein- und Ausgaben. Enthält Werte für Einkommen und Ausgaben und die Berechnung einer Bilanz. \\ \hline
	Cart & Klasse stellt die Umsetzung eines Einkaufswagens dar. \\ \hline
	Order & Klasse stellt die Umsetzung einer Bestellung dar. \\ \hline
	BillViewer & Klasse liefert die Funktionalitäten des Einsehens einer Rechnung für den Kunden \\ \hline
	Bill & Klasse stellt die Umsetzung einer Rechnung dar. \\ \hline
	Account	 & Klasse stellt die Umsetzung eines Personenkontos dar. \\ \hline
	UserManagement  & Ist die Oberklasse zur Verwaltung/Funktionalität des Profils/Kontos. Sie regelt also das Interagieren mit dem Account für den Kunden. Der UserManager benutzt die Funktionalitäten dieser Klasse \\ \hline
	ProfileManagement & Klasse verwaltet die Profile und bietet Funktionalitäten für jeden Nutzer. \\ \hline
	RegisterManagement & Klasse bietet Funktionalitäten zum Registrieren eines jeden Nutzers. \\ \hline
	ProfileViewer & Klasse liefert Funktionalitäten des Einsehens eines Profils für den Benutzer. \\ \hline
	LogInOut & Klasse kümmert sich um die Funktionen des Ein- und Ausloggens. \\ \hline
	ArticleManagement & Die Oberklasse bietet Funktionalitäten zur Verwaltung sämtlicher Artikel. Diese bietet Funktionalitäten zum Durchsuchen der Artikelbestände, sowie zum verwalten dieser. Sie wird vom ArticleManager benutzt. \\ \hline
	CategoryManagement & Klasse kümmert sich um die Kategorisierungs-/Verwaltungsfunktionalitäten jeglicher Artikel. \\ \hline
	BookManagement & Klasse erbt von ArticleManagement und  bietet  Kategorisierungs-/Verwaltungsfunktionalitäten von Büchern \\ \hline
	DVDManagement & Klasse erbt von ArticleManagement und  bietet  Kategorisierungs-/Verwaltungsfunktionalitäten von DVDs \\ \hline
	CDManagement & Klasse erbt von ArticleManagement und  bietet  Kategorisierungs-/Verwaltungsfunktionalitäten von CDs \\ \hline
	Storage & Klasse stellt Umsetzung eines Lagers dar. \\ \hline
	ArticleViewer & Klasse liefert Funktionalitäten zum Anzeigen eines jeden Artikels. \\ \hline
	BookViewer & Klasse erbt von ArticleViewer und liefert Funktionalitäten zum Anzeigen eines Buches. \\ \hline
	DVDViewer & Klasse erbt von ArticleViewer und liefert Funktionalitäten zum Anzeigen einer DVD. \\ \hline
	CDViewer & Klasse erbt von ArticleViewer und liefert Funktionalitäten zum Anzeigen einer CD. \\ \hline
	Article & Klasse stellt die Umsetzung eines Artikels dar. Enthält Werte für den Titel, Cover und eine Beschreibung. \\ \hline
	Category & Klasse stellt die Umsetzung einer Kategorie dar. \\ \hline
	Book & Klasse erbt von Article und stellt die Umsetzung eines Buches dar. Enthält Werte für den Autor, Verlag und die ISBN-Nummer. \\ \hline
	DVD & Klasse erbt von Article und stellt die Umsetzung einer DVD dar. Enthält Werte für Regisseur \\ \hline
	CD & Klasse erbt von Article und stellt die Umsetzung einer CD dar. Enthält Werte für Interpret. \\ \hline
	Genre <Enumeration> & Enumeration ist Teil einer Kategorienklasse und listet alle Genres auf. Enthält die Werte Fiktion, Sachbuch, Unterhaltung, Ratgeber. \\ \hline
	ReadingManagement & Oberklasse bietet Funktionalitäten zur Verwaltung und Einsicht von Veranstaltungen. \\ \hline
	RoomManagement & Klasse kümmert sich um die Verwaltung von Räumlichkeiten. \\ \hline
	CalenderManagement & Klasse verwaltet die Organisation von Veranstaltungen in einem Kalender. \\ \hline
	CalenderViewer & Klasse liefert Einsicht in den Kalender für alle Nutzer \\ \hline
	Room & Klasse stellt die Umsetzung eines Raumes dar. Enthält Werte für Namen und Nummer des Raumes. \\ \hline
	Calender & Klasse stellt die Umsetzung eines Kalenders dar. \\ \hline
	Event & Klasse stellt die Umsetzung einer Veranstaltung dar. Enthält Werte für Namen der Veranstaltung und dessen Datum. \\ \hline
\end{longtable}

\subsection{Persistenz}

\paragraph{Die Daten werden alle in einzelnen Crud Repositories gespeichert. Die Validierung erfolgt im auf der Website über eine Formvalidation Suite, welche durch das Framework ZURB Foundation bereitgestellt wird. Die eingegebenen Daten werden dann in einer zweiten Ebene im Java Backend validiert und notfalls zurückgewiesen. Sind die Daten korrekt, werden sie in ihre jeweiligen Datensätze gespeichert. Die Attribute werden über Setter gesetzt und dann mit repo.save() gespeichert. Beim Start der Applikation werden einige Beispieldatensätze initialisiert. Diese dienen sowohl der Befüllung, Tests als auch der Veranschaulichung}

\subsection{Benutzeroberfläche}

\subsubsection{Überblick: Dialoglandkarte}

\paragraph{Dieser Entwurf zeigt den Anfangsbildschirm eines Nutzers. Von dieser Maske aus kann der Nutzer zu jeder wichtigen Funktionalität der Anwendung gelangen. Daher ist diese Maske am besten geeignet, einen Überblick über die Oberfläche der Anwendung zu schaffen.\\
Auf der linken Seite ist die Sortierung der Artikel zu beeinflussen, am oberen Teil kann die Sortierung durch eine spezielle Suche noch weiter eingegrenzt werden. Außerdem kann ein Nutzer von dieser Seite aus zu dem Kalender gelangen, welchen man auf dem rechten oberen Bildschirmteil auswählen kann. Links daneben befindet sich das Avatarsymbol durch das man ein Drop-Down Menu öffnet, welches die nötigen Optionen zur Verwaltung des Profils bietet. Hier kann man zum Bildschirm für das Ein- und Ausloggen, bzw. auf den Bildschirm für das Registieren gelangen. Außerdem kann man direkt zu seinem Einkaufswagen oder den getätigten Bestellungen gelangen. \\
Unter der Suche befindet sich der Pfad zur Orientierung, um zu Wissen über welchen Weg man zu der aktuellen Seite gelangt ist.
Am unteren Bildschirmrand befindet sich eine Fusszeile, die einen direkten Weg bietet, eine bestimmte Seite der Anwendung zu erreichen.
Der mittige Teil der Oberfläche, welcher in diesem Fall die kategorisierte Bücheraufzählung in einer Tabelle darstellt, wird auf jeder Maske der Anwendung den inhaltlichen Teil darstellen(Artikelsuche, Profildarstellung, Kaufverwaltung, Kalenderdarstellung und das Ein- und Ausloggen, sowie das Registrieren.)\\ \\}
\includegraphics[width=350px]{1Home_Costumer.png}


\subsubsection{Dialogbeschreibung}

\paragraph{Dieser, für die Dialogübersicht bereits verwendete Dialog ist Startseite eines jeden Nutzer, sowie Artikelübersicht mit Kauffunktion.\\ \\}
\includegraphics[width=350px]{1Home_Costumer.png}

\paragraph{Die Maske zeigt das Einloggen eines Nutzers. Es enthält auch die Option zur Registrierung eines noch nicht registrierten Nutzers.\\ \\}
\includegraphics[width=350px]{2Login.png}

\paragraph{Ein Nicht-registrierter Nutzer kann sich über dieses Formular mit einem Klick auf den Knopf Register registrieren, soweit alle Textfelder valide ausgefüllt wurden.\\ \\}
\includegraphics[width=350px]{3Register.png}

\paragraph{Diese Ansicht zeigt die Profilseite eines eingeloggten Kunden. Dieser kann seine angegebenen Daten einsehen und bei Bedarf ändern. Außerdem kann er Rechnungen seiner getätigten Bestellungen einsehen.\\ \\}
\includegraphics[width=350px]{4ProfileView.png}

\paragraph{Das Layout einer Informationsseite für einen Artikel ist ähnlich der eines Nutzerprofils aufgebaut und zeigt die Attribute eines Artikels. Hier kann man sich auch für den Kauf dieses Artikels entscheiden.\\ \\}
\includegraphics[width=350px]{5ArticleView.png}

\paragraph{Diese Maske bietet die Ansicht des Kalenders. Hier können Veranstaltungstermine eingetragen sein, welche der Kunde einsehen kann.\\ \\}
\includegraphics[width=350px]{6CalenderView.png}

\paragraph{Diese Seite zeigt die Artikelsortierung nach dem Artikel Buch spezialisiert, welche durch einen Klick auf die linke Navigationsleiste sortiert wurde. Hier ist es möglich die Bücher in der rechten Tabellenspalte mit einem Klick auf die runden Knöpfe zu selektieren und durch den Knopf kaufen zu der Einkaufswagenansicht mit der Auswahl an Artiken zu wechseln.\\ \\}
\includegraphics[width=350px]{7bookSearch.png}

\paragraph{Dies ist die Einkaufswagenansicht. Hier kann sich der Kunde über die Auswahl seiner Artikel, welche er im Begriff ist zu kaufen vergewissern. Er kann Artikel nach belieben wieder aus dem Einkaufswagen entfernen, nach dem gleichen Layout, wie er auch Artikel selektieren konnte. Mit einem Klick auf den Checkout Knopf gelangt der Kunde zum Kaufsbildschirm.\\ \\}
\includegraphics[width=350px]{8CartView.png}

\paragraph{Hier wird eine Übersicht der Informationen angezeigt, die für die Bestellung eines Artikels von nöten sind. Es werden die selektierten Artikel, sowie die Benutzerinformationen angezeigt, an dessen Adresse der Artikel geschickt werden soll. Wenn alles zur Zufriedenheit des Kunden eingetragen ist, kann dieser auf den Buy Knopf klicken, um die Bestellung abzuschließen.\\ \\}
\includegraphics[width=350px]{9CheckOut.png}

\paragraph{Dieser Bildschirm stellt die Bestätigung einer vom Kunden getätigten Bestellung dar. Von diesem Bild aus kann er wieder auf die Bestellungen zugreifen.\\ \\}
\includegraphics[width=350px]{10SuccessfullyOrdered.png}

\paragraph{Die Einsicht der getätigten Bestellung stellt diese Ansicht dar. Hier kann der Kunde direkt eine Rechnung einsehen und auch die Bestellung wieder stornieren.\\ \\}
\includegraphics[width=350px]{11OrderView.png}

\paragraph{Diese Maske zeigt die Einsicht einer Rechnung. Der Kunde kann eine Rechnung auch als Dokument anfordern.\\ \\}
\includegraphics[width=350px]{12BillView.png}

\paragraph{Bei einer Stornierung der Bestellung wird ein letztes zu bestätigendes Bild angezeigt, um eine Bestellung rückgängig zu machen.\\ \\}
\includegraphics[width=350px]{13Cancelation.png}

\paragraph{Diese Ansicht stellt den üblichen Startbildschirm eines Angestellten dar. Diese unterscheidet sich nur aufgrund der rechten Navigationsleiste grundlegend von der des Kunden bzw. allgemeinen Nutzers. Diese Navigation zeigt die Verwaltungsoptionen, die ein Angestellter angepasst auf seine Rolle bzw. Befugnis ausführen kann.\\
In diesem Beispiel handelt es sich um den Bildschirm eines Artikelverwalters, da auf dieser Seite bereits die Verwaltung der Artikel angedeutet ist.\\ \\}
\includegraphics[width=350px]{14Home_Employee.png}

\paragraph{Diese Maske zeigt die Ansicht eines Benutzerverwalters, der nach Profilen suchen, sowie diese verwalten kann.\\ \\}
\includegraphics[width=350px]{15ChangeUser.png}

\paragraph{Hier wird die Option eines Artikelverwalters angezeigt, die Genres zu verwalten, die einem Artikel angehören können.\\ \\}
\includegraphics[width=350px]{16ChangeGenre.png}

\paragraph{Das Verwalten eines speziellen Artikels zeigt diese Maske, wieder aus der Sicht eines Artikelverwalters.\\ \\}
\includegraphics[width=350px]{17ChangeArticle.png}

\paragraph{Der Veranstaltungsverwalter kann in dieser Ansicht Kalender, sowie Veranstaltungen und Räume verwalten.\\ \\}
\includegraphics[width=350px]{18ChangeCalender.png}

\paragraph{Zur Raumverwaltung kommt der Veranstaltungsverwalter über die Kalenderverwaltungsansicht. Hier können Räume für Veranstaltungen, sowie für die Buchhandlung im allgemeinen verwaltet werden.\\ \\}
\includegraphics[width=350px]{19ChangeRoom.png}

\subsection{Ergonomie}

\paragraph{Das Crosstesting wurde durch die Gruppe 16 durchgeführt und ein Testbericht angefertigt. Die dort aufgeführten Bugs wurden untersucht und in vollem Umfang behoben. Um den höchsten Standards gerecht zu werden, wurden auch sogenannte Omatests durchgeführt. Die Applikation wurde verschiedenen Zielgruppen vorgelegt. Als Erstes den Eltern eines Teammitglieds und als Zweites den Mitbewohnern eines anderen Teammitglieds.}

\subsection{Transaktions- und Sessionbehandlung}

\paragraph{Die Nutzer interagiert mit der Anwendung nur über die Weboberfläche. Es gibt zwei grunsätzliche Methoden. Diese sind \textit{GET} und \textit{POST}. Der Nutzer sendet alle seine Anfragen über diese Methoden an den Server. Größtenteils enthält diese Anwendung Formulare, welche per \textit{POST} übermittelt werden. Es gibt aber auch Requests welche über \textit{GET} gemappt wurden. Der Server behandelt diese Anfragen über die jeweiligen Controller und und schickt die Daten an die Model, welche dann Ihre Zustände ändern. Jeder Nutzer kann sich auch einen Benutzeraccount anlegen, welcher ihm authentifiziert und weitere Aktionen ermöglicht.}

\subsection{Sicherheit}

\paragraph{Spring Security wurde umfassend und erschöpfend verwendet. Näheres zu den Funktionen ist unter \url{http://projects.spring.io/spring-security/} zu finden.}

\subsection{Kommunikation und Integration mit anderen IT-Systemen}

\paragraph{Ein Apache lässt sich mit ProxyPassReverse als vorgeschaltete Instanz zum Web betreiben. Dadurch wird eine Portspezifizierung ":8080" überflüssig.}

\subsection{Verteilung}

\paragraph{Diese Anwedung ist eigenständig und läuft nur auf einem Server.}

\subsection{Plausibilisierung und Validierung}

\paragraph{Jegliche Benutzereingaben werden sowohl durch Javascript als auch durch das Java-Backend validiert.}

\subsection{Ausnahme-/Fehlerbehandlung}

\paragraph{Es wurde eine Standardfehlerseite implementiert. Eine weitere Asuwertung ist zumindest für den normalen Websitebesucher nicht sinnvoll. Für Administratoren steht das Serverlog zur Verfügung.}

\subsection{Logging, Protokollierung, Tracing}

\begin{itemize}
	\item Serverlog
\end{itemize}

\subsection{Konfigurierbarkeit}

\paragraph{Es können neue Nodes hinzugefügt, geändert und gelöscht werden. Im Datainitializer werden die Grundkonfigurationen für die Anwendung vorgenommen.}

\subsection{Internationalisierung}

\paragraph{Internationalisierung ist konfigurierbar durch die messages.properties.}

\subsection{Testbarkeit}

\begin{longtable}{|p{50px}|p{200px}|p{100px}|}
	\hline
	\rowcolor[HTML]{C0C0C0} 
	Test Case ID & Beschreibung & Ergebnis \\ \hline
	TC00000 & Versuche etwas zu tun ohne die notwendigen Nutzerrechte zu besitzen & Fehlermeldung \\ \hline
	TCUC011 & Eingabe des korrekten Nutzernamens und Passwortes & Login und Startseite \\ \hline
	TCUC012 & Eingabe des falschen Nutzernamens & Login-Fehler und erneute Eingabe \\ \hline
	TCUC013 & Eingabe des falschen Passwortes & Login-Fehler und erneute Eingabe \\ \hline
	TCUC021 & Eingabe von korrekten und nicht vergebenen Registrierungsdaten & Registrierung als Nutzer \\ \hline
	TCUC022 & Eingabe eines bereits verwendeten Benutzernamens & Fehlermeldung und erneute Eingabe \\ \hline
	TCUC023 & Eingabe eines falschen Adressformates (E-Mail oder Lieferadresse) & Fehlermeldung und erneute Eingabe \\ \hline
	TCUC031 & Änderung der Rolle eines Nutzers & neue Zugriffsrechte des Nutzers \\ \hline
	TCUC041 & Eingabe von korrekten Nutzerdaten & Nutzerdaten geändert \\ \hline
	TCUC042 & Eingabe eines falschen Adressformates (E-Mail oder Lieferadresse) & Fehlermeldung und erneute Eingabe \\ \hline
	TCUC051 & Öffnen der Profilansicht & korrekte Nutzerdaten \\ \hline
	TCUC061 & Öffnen der Profilansicht & korrekte Nutzerdaten \\ \hline
	TCUC071 & korrektes Zurücksetzen des Passwortes & neues Passwortes versandt \\ \hline
	TCUC071 & Eingabe eines nichtexistenten Nutzernamens & Fehlermeldung \\ \hline
	TCUC081 & korrektes Löschen des Nutzerprofils & Profil gelöscht \\ \hline
	TCUC082 & Löschen des letzten Administrators & Fehlermeldung \\ \hline
	TCUC083 & Löschen eines nichtexistenten Nutzers & Fehlermeldung \\ \hline
	TCUC091 & korrektes Sperren des Nutzerprofils & Profil gesperrt \\ \hline
	TCUC092 & Sperren eines Administrators & Fehlermeldung \\ \hline
	TCUC093 & Sperren eines nichtexistenten Nutzers & Fehlermeldung \\ \hline
	TCUC101 & Öffnen der Artikelansicht & korrekte Artikeldaten \\ \hline
	TCUC111 & Eingabe von korrekten Artikeldaten & Artikel hinzugefügt \\ \hline
	TCUC112 & Eingabe eines falschen Formates & Fehlermeldung und erneute Eingabe \\ \hline
	TCUC113 & Eingabe von bereits existierenden Artikeldaten & Fehlermeldung und erneute Eingabe \\ \hline
	TCUC121 & korrektes Entfernen des Artikels & Artikel entfernt \\ \hline
	TCUC122 & Entfernen eines nicht vorhandenen Artikels & Fehlermeldung \\ \hline
	TCUC131 & korrektes Durchsuchen der Artikel nach Kriterien & Anzeige der Artikelliste \\ \hline
	TCUC132 & Eingabe von Artikeldaten zu denen kein Artikel existiert & leere Artikelliste \\ \hline
	TCUC141 & Eingabe von korrekten Artikeldaten & Artikeldaten geändert \\ \hline
	TCUC142 & Eingabe eines falschen Formates & Fehlermeldung und erneute Eingabe \\ \hline
	TCUC143 & Eingabe von bereits existierenden Artikeldaten & Fehlermeldung und erneute Eingabe \\ \hline
	TCUC151 & Eingabe von korrekten und nicht vergebenen Daten & Kategorie geändert \\ \hline
	TCUC152 & Eingabe eines falschen Formates & Fehlermeldung und erneute Eingabe \\ \hline
	TCUC161 & Eingabe von korrekten und nicht vergebenen Daten & Kategorie hinzugefügt \\ \hline
	TCUC162 & Eingabe eines falschen Formates & Fehlermeldung und erneute Eingabe \\ \hline
	TCUC171 & korrektes Entfernen der Kategorie & Kategorie entfernt und Artikel aus ihr entfernt \\ \hline
	TCUC172 & Entfernen einer nicht vorhandenen Kategorie & Fehlermeldung \\ \hline
	TCUC181 & korrektes Hinzufügen des Artikel zur Kategorie & Artikel gehört zur Kategorie \\ \hline
	TCUC182 & Hinzufügen eines nicht vorhandenen Artikels & Fehlermeldung \\ \hline
	TCUC183 & Hinzufügen zu einer nicht vorhandenen Kategorie & Fehlermeldung \\ \hline
	TCUC191 & korrektes Entfernen des Artikel von der Kategorie & Artikel aus Kategorie entfernt \\ \hline
	TCUC192 & Entfernen eines nicht vorhandenen Artikels & Fehlermeldung \\ \hline
	TCUC193 & Entfernen aus Kategorie, obwohl Artikel nicht zur Kategorie gehört & Fehlermeldung \\ \hline
	TCUC201 & Öffnen der Warenkorbansicht & korrekter Warenkorbinhalt \\ \hline
	TCUC211 & korrektes Hinzufügen des Artikels zum Warenkorb & Artikel im Warenkorb \\ \hline
	TCUC212 & Hinzufügen eines nicht vorhandenen Artikels zum Warenkorb & Fehlermeldung \\ \hline
	TCUC221 & korrektes Entfernen des Artikels vom Warenkorb & Artikel aus dem Warenkorb entfernt \\ \hline
	TCUC222 & Entfernen eines nicht nicht im Warenkorb liegenden Artikels vom Warenkorb & Fehlermeldung \\ \hline
	TCUC223 & Entfernen eines nicht vorhandenen Artikels vom Warenkorb & Fehlermeldung \\ \hline
	TCUC231 & korrektes Kaufen der Artikel & Bestellung hinzugefügt, Rechnung erstellt und versandt \\ \hline
	TCUC232 & Kaufen eines nicht vorhandenen Artikels & Fehlermeldung \\ \hline
	TCUC233 & Kaufen als nicht eingeloggter Gast & Fehlermeldung und LogIn-Seite \\ \hline
	TCUC241 & Öffnen der Bestellungsansicht & Liste aller Bestellungen \\ \hline
	TCUC251 & korrektes Abbrechen der Bestellung & Bestellung abgebrochen \\ \hline
	TCUC252 & Abbrechen einer nicht vorhandenen Bestellung & Fehlermeldung \\ \hline	
	TCUC253 & Abbrechen einer bereits versandten Bestellung & Fehlermeldung \\ \hline
	TCUC261 & korrektes Ausgeben des Retourenscheins  & Retourenschein versandt \\ \hline
	TCUC271 & Öffnen der Rechnungsansicht & Anzeige der Rechnung \\ \hline	
	TCUC281 & Öffnen der Ansicht der Statistiken mit Admin- oder Chef-Rechten & Anzeige der Statistiken \\ \hline
	TCUC291 & Öffnen der Kalenderansicht & Anzeige des Kalenders \\ \hline
	TCUC301 & Eingabe von korrekten und nicht vergebenen Daten & Lesung hinzugefügt \\ \hline
	TCUC302 & Eingabe eines falschen Formates & Fehlermeldung und erneute Eingabe \\ \hline
	TCUC303 & Eingabe von vergebenem Raum und vergebener Zeit & Fehlermeldung und erneute Eingabe \\ \hline
	TCUC311 & Eingabe von korrekten und nicht vergebenen Daten & Lesung geändert \\ \hline
	TCUC312 & Eingabe eines falschen Formates & Fehlermeldung und erneute Eingabe \\ \hline
	TCUC313 & Eingabe von vergebenem Raum und vergebener Zeit & Fehlermeldung und erneute Eingabe \\ \hline
	TCUC321 & korrektes Entfernen der Lesung & Lesung entfernt \\ \hline
	TCUC322 & Entfernen einer nicht vorhandenen Lesung & Fehlermeldung \\ \hline	
	TCUC331 & Eingabe von korrekten und nicht vergebenen Daten & Raum hinzugefügt \\ \hline
	TCUC332 & Eingabe eines falschen Formates & Fehlermeldung und erneute Eingabe \\ \hline
	TCUC333 & Eingabe von vergebenem Raumnamen & Fehlermeldung und erneute Eingabe \\ \hline
	TCUC341 & Eingabe von korrekten und nicht vergebenen Daten & Raum geändert \\ \hline
	TCUC342 & Eingabe eines falschen Formates & Fehlermeldung und erneute Eingabe \\ \hline
	TCUC343 & Eingabe von vergebenem Raumnamen & Fehlermeldung und erneute Eingabe \\ \hline
	TCUC351 & korrektes Entfernen des Raumes & Raum entfernt \\ \hline
	TCUC352 & Entfernen eines nicht vorhandenen Raumes & Fehlermeldung \\ \hline
	TCUC353 & Entfernen eines Raumes mit zukünftigen Lesungen & Fehlermeldung \\ \hline
\end{longtable}

\subsection{Buildmanagement}

\paragraph{Das Github Repository bookshop-schiller beinhaltet die komplette Applikation. Die pom.xml für Maven befindet sich im Rootverzeichnis der Anwendung. Über diese werden die Dependencies durch Maven gemanaged. Travis CI buildet jeden Commit und gibt Warnungen oder Fehlermeldungen aus.}

\end{document}