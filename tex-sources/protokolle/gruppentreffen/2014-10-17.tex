\documentclass[12pt,a4paper]{article}

\usepackage[utf8]{inputenc}
\usepackage[english,german]{babel}
\usepackage{amsmath}
\usepackage{amsfonts}
\usepackage{amssymb}
\usepackage{amsthm}
\usepackage[left=2cm,right=2cm,top=2cm,bottom=2cm]{geometry}
\setlength{\parindent}{0pt}

\author{Protokollführer: keiner; Überarbeitet von: Till Köhler}
\title{Sitzungsprotokoll Gruppentreffen 2014-10-17}
\date{}

\begin{document}

\maketitle

\subsection*{Anwesenheit}
\medskip
\begin{itemize}
\item Philipp Waack (Chefprogrammierer)
\item Philipp Jäschke (Assistent)
\item Till Köhler (Testverantwortlicher)
\end{itemize}

\subsection*{Dauer des Treffens}
\medskip
\begin{itemize}
\item Beginn: 17:15 Uhr
\item Ende: 19:15 Uhr
\end{itemize}

\noindent\rule{\textwidth}{1pt}

\subsection*{Tagespunkte}
\medskip

\subsubsection*{Allgemeine organisatorische Absprachen}
\begin{itemize}
\item Wiederholung der Einführungsveranstaltung
\item Klärung: Welche Rolle ist für was genau zuständig?
\item Klärung: Was ist in den nächsten Tagen zu tun? (siehe unten)
\item Beschluss: Lieber Vorarbeiten, als später in Stress zu geraten!
\end{itemize}

\subsubsection*{Termin für gemeinsame Gruppentreffen}
\begin{itemize}
\item ab sofort: wöchentlich, Montag 13:00 - 15:00 Uhr (Treff: Foyer INF-Bau)
\end{itemize}

\subsubsection*{FusionForge}
\begin{itemize}
\item Klärung: Was ist FusionForge? Was kann man damit machen?
\item Ansehen der Benutzeroberfläche (SCM, Wiki, ...)
\end{itemize}

\subsubsection*{Gemeinsame Einrichtung der erforderlichen Software}
\begin{itemize}
\item Vorgehensweise, anhand des Guestbook-Beispieles
\item Download/Installation von:
\begin{itemize}
\item Spring Tool Suite
\item Maven
\item Git
\end{itemize}
\end{itemize}

\subsubsection*{Aufgaben bis zum nächsten Gruppentreffen}
\begin{itemize}
\item Philipp W.: Grobentwurf Projektplan (in Absprache mit Philipp J.)
\item Chris: Aufsetzen der Standard-Website; Einrichten des MediaWikis im FusionForge
\item Alle: Fertige Einrichtung der Software; Download und Ansehen des Guestbook-Repos; Grobentwurf zum Use-Case-Diagramm
\end{itemize}

\end{document}