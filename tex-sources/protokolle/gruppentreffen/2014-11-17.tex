\documentclass[12pt,a4paper]{article}

\usepackage[utf8]{inputenc}
\usepackage[english,german]{babel}
\usepackage{amsmath}
\usepackage{amsfonts}
\usepackage{amssymb}
\usepackage{amsthm}
\usepackage[left=2cm,right=2cm,top=2cm,bottom=2cm]{geometry}
\setlength{\parindent}{0pt}

\author{Protokollführer: Till Köhler}
\title{Sitzungsprotokoll Gruppentreffen 2014-11-17}
\date{}

\begin{document}

\maketitle

\subsection*{Anwesenheit}
\medskip
\begin{itemize}
\item Philipp Waack (Chefprogrammierer)
\item Philipp Jäschke (Assistent)
\item Christoph Kepler (Operator)
\item Maximilian Dühr (Testverantwortlicher)
\item Till Köhler (Sekretär)
\end{itemize}

\subsection*{Dauer des Treffens}
\medskip
\begin{itemize}
\item Beginn: 13:00 Uhr
\item Ende: 14:15 Uhr
\end{itemize}

\noindent\rule{\textwidth}{1pt}

\subsection*{Tagespunkte}
\medskip

\subsubsection*{Vorstellung des Entwurf-Klassendiagramm}
(durch Philipp W. und Max)
\begin{itemize}
\item Frage: Wie bindet man korrekt externe Klassen (aus dem Salespoint-Framework) in das Klassendiagramm ein?
\item verstärktes Umsetzen von Singleton-Entwurfsmustern (z.B. bei den Manager-Klassen)
\end{itemize}

\subsubsection*{Vorstellung des GUI-Prototypen}
(durch Chris)
\begin{itemize}
\item GUI-Prototypen bereits fertiggestellt, aber es gab einen Speicherfehler
\item GUI-Prototyp muss komplett neu geschrieben werden
\item Vorstellung erfolgt dennoch mit Papier und Stift
\item Website auf gesamte Bildschirmbreite vergrößern
\end{itemize}

\subsubsection*{Vorstellung der Akzeptanzfälle im Pflichtenheft}
(durch Till)
\begin{itemize}
\item Anwendungsfälle wurden ebenfalls mit überarbeitet (wegen Konsistenz der Nummerierung von Anwendungs- und Akzeptanzfällen)
\end{itemize}

\subsubsection*{Absprache zum Tutorentreffen 2014-11-17}
\begin{itemize}
\item Frage: Findet das Tutorentreffen trotz Feiertag statt?
\item Philipp W. schreibt E-Mail an Tutor
\end{itemize}

\subsubsection*{Aufgaben bis zum Tutorentreffen 2014-11-19}
\begin{itemize}
\item Philipp W. und Max: Überarbeiten des Entwurfs-Klassendiagrammes
\item Chris: erneute Fertigstellung des GUI-Prototypen
\item Alle: Überlegungen zu Anwendungsprototypen: Welche Funktionen sollen realisiert werden? Wie ist das umsetzbar?
\end{itemize}

\subsubsection*{Aufgaben bis nächste Woche}
\begin{itemize}
\item Alle: Fertigstellung der Anwendungsprototypen; gemeinsame Vorbereitung und Fertigstellung der Zwischenpräsentation
\end{itemize}

\end{document}