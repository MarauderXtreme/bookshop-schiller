\documentclass[12pt,a4paper]{article}

\usepackage[utf8]{inputenc}
\usepackage[english,german]{babel}
\usepackage{amsmath}
\usepackage{amsfonts}
\usepackage{amssymb}
\usepackage{amsthm}
\usepackage[left=2cm,right=2cm,top=2cm,bottom=2cm]{geometry}
\setlength{\parindent}{0pt}

\author{Protokollführer: Till Köhler}
\title{Sitzungsprotokoll Gruppentreffen 2014-11-03}
\date{}

\begin{document}

\maketitle

\subsection*{Anwesenheit}
\medskip
\begin{itemize}
\item Philipp Waack (Chefprogrammierer)
\item Philipp Jäschke (Assistent)
\item Christoph Kepler (Operator)
\item Maximilian Dühr (Testverantwortlicher)
\item Till Köhler (Sekretär)
\end{itemize}

\subsection*{Dauer des Treffens}
\medskip
\begin{itemize}
\item Beginn: 13:00 Uhr
\item Ende: 15:00 Uhr
\end{itemize}

\noindent\rule{\textwidth}{1pt}

\subsection*{Tagespunkte}
\medskip

\subsubsection*{Vorstellung des Pflichtenheftes}
(durch Chris)
\begin{itemize}
\item Struktur des Pflichtenheftes: super
\item Frage: Was genau gehört in den ''fachlichen Überblick''?
\item Ergänzen vom Storyboard 
\item Frage: Genauere Beschreibung für Diagramme nötig oder nur als Bilder einfügen?
\item Trennen des Use-Case-Diagrammes zum besseren Einfügen in das Dokument
\item Ergänzung der Sequenzdiagramme, sobald sie fertig sind (erstellt von Philipp J.)
\item Ergänzung der Anwendungsfallbeschreibungen
\item Ergänzung der Beschreibungen zu den Klassen ergänzen (macht Philipp W.)
\item Ergänzung aller ungeklärten Fragen in ''Offene Punkte''
\item Mögliche weitere Punkte: Qualitätsanforderungen, Glossar, Technische Anforderungen
\item Pflichtenheft muss permanent aktuell gehalten werden (!)
\end{itemize}

\subsubsection*{Vorstellung der Sequenzdiagramme}
(durch Philipp J.)
\begin{itemize}
\item Kleinigkeiten direkt ausgebessert
\item Hinzufügen der Stornierung eines Artikels
\item Anmerkung ergänzen, dass ''Book'' durch beliebigen Artikel (CD, DVD, ...) ergänzt werden kann
\item Hinzufügen einer Änderung des eigenen Profils
\item Änderung der Benennung der Sequenzdiagramme
\end{itemize}

\subsubsection*{Vorstellung des Storyboards}
(durch Philipp W. und Max)
\begin{itemize}
\item kein Profilbildes für Kunden nötig
\item Lesungen im Kalender farbig hervorheben
\item Einfügen einer Möglichkeit eine Lesung im Kalender auszuwählen (Aufploppen der Informationen zur Lesung)
\item Design muss noch überarbeitet werden (in späterer Phase des Projektes)
\end{itemize}

\subsubsection*{Vorstellung des Analyse-Klassendiagrammes}
(durch Philipp W. und Max)
\begin{itemize}
\item Ausbessern von Kleinigkeiten
\item Konsistenz zum Use-Case-Diagramm (bezüglich Bezeichnungen)
\item Einbauen genauer Salespoint-Klassen
\end{itemize}

\subsubsection*{Absprache zu den Prototypen}
\begin{itemize}
\item Frage: Was muss dazu genau gemacht werden?
\item Jeder kümmert sich um einen bestimmten Teil des Prototypen
\item Aufgabenverteilung findet später statt
\end{itemize}

\subsubsection*{Aufgaben bis zum Tutorentreffen 2014-11-05}
\begin{itemize}
\item Chris: Überarbeitung des Pflichtenheftes
\item Philipp J.: Überarbeitung des Sequenzdiagramme
\item Philipp W. und Max: Überarbeiten des Analyse-Klassendiagrammes und des Storyboards
\item Alle: Genaues Ansehen das Analyse-Klassendiagrammes (Was fehlt? Was ist falsch?)
\end{itemize}

\subsubsection*{Aufgaben bis nächste Woche}
\begin{itemize}
\item Philipp W. und Max: Erstellung der Enfwurfs-Klassendiagrammes
\item Chris: Erstellen der GUI der Anwendung in HTML
\item Till: Tiefere Einarbeitung in das Salespoint-Framework, Hilfe beim Entwurfs-Klassendiagramm bezüglich verwendbarer Salespoint-Klassen und -Interfaces
\item Philipp J.: Hilfe bei der Erstellung der GUI in HTML
\end{itemize}

\end{document}