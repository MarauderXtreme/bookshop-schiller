\documentclass[12pt,a4paper]{article}

\usepackage[utf8]{inputenc}
\usepackage[english,german]{babel}
\usepackage{amsmath}
\usepackage{amsfonts}
\usepackage{amssymb}
\usepackage{amsthm}
\usepackage[left=2cm,right=2cm,top=2cm,bottom=2cm]{geometry}
\setlength{\parindent}{0pt}

\author{Protokollführer: Till Köhler}
\title{Sitzungsprotokoll Gruppentreffen 2014-12-08}
\date{}

\begin{document}

\maketitle

\subsection*{Anwesenheit}
\medskip
\begin{itemize}
\item Philipp Waack (Chefprogrammierer)
\item Philipp Jäschke (Assistent)
\item Christoph Kepler (Operator)
\item Till Köhler (Sekretär)
\end{itemize}

\subsection*{Dauer des Treffens}
\medskip
\begin{itemize}
\item Beginn: 13:00 Uhr
\item Ende: 14:45 Uhr
\end{itemize}

\noindent\rule{\textwidth}{1pt}

\subsection*{Tagespunkte}
\medskip

\subsubsection*{Abwesenheit von Max}
\begin{itemize}
\item entschuldigt
\end{itemize}

\subsubsection*{Vorstellung der einzelnen Arbeitsfortschritte}
\begin{itemize}
\item Philipp W.: Hinzufügen, Ändern und Löschen von Artikeln; Einfügen von CD-, DVD- und Artikelübersicht; Erhöhung und Verringerung der Anzahl verfügbarer Artikel
\item Philipp J.: Berechnung der Geldsumme im Warenkorb; Löschen von Artikeln aus dem Warenkorb; Probleme mit Thymeleaf-Variablen
\item Till: Überarbeitung der Registrierung von Nutzern; Hinzufügen einer Nutzer-Profilansicht; Hinzufügen von Rollen, Status und Profilverlinkung zur Nutzerübersicht
\item Chris: Überarbeitung des GUI-Designs
\end{itemize}

\subsubsection*{Auswertung der einzelnen Arbeitsfortschritten}
\begin{itemize}
\item Klärung des weiteren Vorhabens
\item Klärung individueller Fragen und Probleme
\end{itemize}

\subsubsection*{Aufgaben bis zum Tutorentreffen 2014-12-10}
\begin{itemize}
\item Alle: individuelle Arbeit am eigenen Implementierungsbereich
\end{itemize}

\subsubsection*{Aufgaben bis nächste Woche}
\begin{itemize}
\item Alle: individuelle Arbeit am eigenen Implementierungsbereich; Anfang von JUnit-Tests, Entwurf-Klassendiagramm, JavaDoc und der Entwicklerdokumentation
\end{itemize}

\end{document}