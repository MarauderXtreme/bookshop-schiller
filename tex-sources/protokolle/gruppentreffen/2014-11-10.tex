\documentclass[12pt,a4paper]{article}

\usepackage[utf8]{inputenc}
\usepackage[english,german]{babel}
\usepackage{amsmath}
\usepackage{amsfonts}
\usepackage{amssymb}
\usepackage{amsthm}
\usepackage[left=2cm,right=2cm,top=2cm,bottom=2cm]{geometry}
\setlength{\parindent}{0pt}

\author{Protokollführer: Till Köhler}
\title{Sitzungsprotokoll Gruppentreffen 2014-11-10}
\date{}

\begin{document}

\maketitle

\subsection*{Anwesenheit}
\medskip
\begin{itemize}
\item Philipp Waack (Chefprogrammierer)
\item Philipp Jäschke (Assistent)
\item Christoph Kepler (Operator)
\item Maximilian Dühr (Testverantwortlicher)
\item Till Köhler (Sekretär)
\end{itemize}

\subsection*{Dauer des Treffens}
\medskip
\begin{itemize}
\item Beginn: 13:15 Uhr
\item Ende: 14:45 Uhr
\end{itemize}

\noindent\rule{\textwidth}{1pt}

\subsection*{Tagespunkte}
\medskip

\subsubsection*{Zur Ordnerstruktur im GitHub}
\begin{itemize}
\item Dokumente nicht doppelt ins GitHub hochladen (alte Version löschen und Benennung beibehalten)
\item halbwegs fertige Dokumente werden nur ins public Repository gepusht 
\item kurz vor der Abgabe wird das private Repository geupdated (durch Till oder Chris)
\item nach Update der UML-Diagramme immer auch Bilddateien updaten
\end{itemize}

\subsubsection*{Absprache zu den Aufgaben zum Videoshop}
\begin{itemize}
\item Klärung von allgemeinen und individuellen Problemen
\end{itemize}

\subsubsection*{Klärung des Terminplans für die Entwurfsphase}
\begin{itemize}
\item siehe Sitzungsprotokoll Tutorentreffen 2014-11-05
\end{itemize}

\subsubsection*{Vorstellung des Pflichtenheftes}
(durch Chris)
\begin{itemize}
\item Design und Struktur sehr gut, da jetzt komplett in \LaTeX
\item Hinzufügen der GUI
\item Frage zu den Testfällen: Nur Negativ-Tests oder auch Positiv-Tests? 
\end{itemize}

\subsubsection*{Vorstellung der Zustandsdiagramme}
(durch Philipp J. und Till)
\begin{itemize}
\item (Till) Book und Reading: Entfernen der Zustände ''Adding'' und ''Removing'' und Umwandeln in Aktionen
\item (Philipp) Login/Registration: ausführlicheres ''Checking''
\end{itemize}

\subsubsection*{Aufgaben bis zum Tutorentreffen 2014-11-12}
\begin{itemize}
\item Chris: Einfügen der GUI ins Pflichtenheft
\item Philipp W. und Max: Beginn mit dem Entwurfs-Klassendiagramm
\item Philipp J.: Überarbeitung des Zustandsdiagrammes Login/Registration
\item Till: Überarbeitung der Zustandsdiagramme Reading und Book
\item Alle: Fertigstellung der Aufgaben zum Programmierbeispiel Videoshop
\end{itemize}

\subsubsection*{Aufgaben bis nächste Woche}
\begin{itemize}
\item Philipp W. und Max: Fertigstellung des Entwurfs-Klassendiagrammes
\item Chris: Entwicklung eines groben GUI-Prototypen in HTML
\end{itemize}

\end{document}