\documentclass[12pt,a4paper]{article}

\usepackage[utf8]{inputenc}
\usepackage[english,german]{babel}
\usepackage{amsmath}
\usepackage{amsfonts}
\usepackage{amssymb}
\usepackage{amsthm}
\usepackage[left=2cm,right=2cm,top=2cm,bottom=2cm]{geometry}
\setlength{\parindent}{0pt}

\author{Protokollführer: Till Köhler}
\title{Sitzungsprotokoll Gruppentreffen 2014-12-01}
\date{}

\begin{document}

\maketitle

\subsection*{Anwesenheit}
\medskip
\begin{itemize}
\item Philipp Waack (Chefprogrammierer)
\item Philipp Jäschke (Assistent)
\item Christoph Kepler (Operator)
\item Maximilian Dühr (Testverantwortlicher)
\item Till Köhler (Sekretär)
\end{itemize}

\subsection*{Dauer des Treffens}
\medskip
\begin{itemize}
\item Beginn: 13:00 Uhr
\item Ende: 14:00 Uhr
\end{itemize}

\noindent\rule{\textwidth}{1pt}

\subsection*{Tagespunkte}
\medskip

\subsubsection*{Vorstellung der einzelnen Arbeitsfortschritte}
\begin{itemize}
\item Till: Registrierung und Benutzerverwaltung erweitert
\end{itemize}

\subsubsection*{Absprache zu noch fehlenden GUI-Funktionalitäten}
\begin{itemize}
\item Einfügen einer Profilansicht, inklusive der Möglichkeit Profildaten zu ändern, das Profil zu löschen, das Passwort zurückzusetzen, ...
\item Einfügen einer Möglichkeit für den Administrator ein neues Angestelltenprofil anzulegen
\item Einfügen einer übersichtlichen Kalenderansicht
\item Einfügen von Ansichten für Bestellungen, Inventar, Statistiken, ... für Boss und Administrator
\item Einfügen von Bestätigungsseiten für das Löschen und Ändern von Profilen, Artikeln, ... 
\end{itemize}

\subsubsection*{Absprache zu Entwurfs-Klassendiagramm und JavaDoc}
\begin{itemize}
\item permanentes Festhalten des eigenen Arbeitsfortschritt im Entwurfsklassendiagramm und mithilfe von JavaDoc im Projektordner
\end{itemize}

\subsubsection*{Aufgaben bis zum Tutorentreffen 2014-12-03}
\begin{itemize}
\item Alle: Einbau kleinerer, vorzeigbarer Funktionalitäten
\item Chris: kleinere Verbesserungen an der GUI
\end{itemize}

\subsubsection*{Aufgaben bis nächste Woche}
\begin{itemize}
\item Alle: weitere Arbeit an der Implementierung des Projektes; Aktualisierung von Entwurfs-Klassendiagramm und JavaDoc
\item Chris: Einbau der fehlenden Funktionalitäten in die GUI; Verbesserung des GUI-Designs
\end{itemize}

\end{document}