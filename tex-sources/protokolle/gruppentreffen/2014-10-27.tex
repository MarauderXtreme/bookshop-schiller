\documentclass[12pt,a4paper]{article}

\usepackage[utf8]{inputenc}
\usepackage[english,german]{babel}
\usepackage{amsmath}
\usepackage{amsfonts}
\usepackage{amssymb}
\usepackage{amsthm}
\usepackage[left=2cm,right=2cm,top=2cm,bottom=2cm]{geometry}
\setlength{\parindent}{0pt}

\author{Protokollführer: Till Köhler}
\title{Sitzungsprotokoll Gruppentreffen 2014-10-27}
\date{}

\begin{document}

\maketitle

\subsection*{Anwesenheit}
\medskip
\begin{itemize}
\item Philipp Waack (Chefprogrammierer)
\item Philipp Jäschke (Assistent)
\item Christoph Kepler (Operator)
\item Maximilian Dühr (Testverantwortlicher)
\item Till Köhler (Sekretär)
\end{itemize}

\subsection*{Dauer des Treffens}
\medskip
\begin{itemize}
\item Beginn: 13:30 Uhr
\item Ende: 15:30 Uhr
\end{itemize}

\noindent\rule{\textwidth}{1pt}

\subsection*{Tagespunkte}
\medskip

\subsubsection*{Neue Rollenverteilung (!)}
\begin{itemize}
\item Testverantworticher: Max (Grund: bessere technische Kenntnisse)
\item Sekretär: Till (Grund: {\LaTeX}-Kenntnisse)
\end{itemize}

\subsubsection*{Installation von MagicDraw}
\begin{itemize}
\item für alle ohne bisherige MagicDraw-Lizenz
\item Installationsdatei und Lizenzschlüssel von Till bereitgestellt
\end{itemize}

\subsubsection*{Vorstellung von Anwendungsfällen und Use-Case-Diagramm}
(durch Till) 
\begin{itemize}
\item Frage: Wer registriert den Administrator? Lösung: Nutzer 0 erhält alle Rechte (automatischer Administrator)
\item Änderung von requestDeleteProfile zu deleteProfile(Grund: direktes Löschen des eigenen Profils muss möglich sein)
\item requestPasswortReset muss durch Guest möglich sein (Grund: Customer ist zu diesem Zeitpunkt nicht eingeloggt)
\item UserManager darf Kontodaten nicht ändern (nur Admin im äußersten Notfall!)
\item Entfernen von resetPasswordOfCustomer (Grund: Automatische, systeminterne Verwaltung)
\item Änderung von sendBackArticle zu printReturnsNote (Grund: das eigentliche Rücksenden des Artikel geschieht nicht über das System, sondern per Retourenschein und Post) 
\item Einführung eines BillManagaments zur Verwaltung der Rechnungen
\item Einführung weiterer Überklassen (?)
\end{itemize}

\subsubsection*{Vorstellung von Top-Level-Architektur und Kontextmodell}
(durch Philipp W. und Max)
\begin{itemize}
\item Konsistenz zum Anwendungsfalldiagramm (!)
\item Überarbeitung auf Basis des überarbeiteten Anwendungsfalldiagrammes durch Philipp W. und Max
\item Frage: Wie sollen Anwendungsfälle ohne Überklassen dargestellt werden?
\end{itemize}

\subsubsection*{Absprache zu den Sequenzdiagrammen}
\begin{itemize}
\item Frage: Welche Szenarien sollen umgesetzt werden? Beispiele?
\item Philipp J. kümmert sich um die Erstellung
\end{itemize}

\subsubsection*{Absprache zur überarbeiteten Website}
\begin{itemize}
\item allgemeine Zustimmung
\item kleine Verbesserungen in der Formatierung noch nötig
\end{itemize}

\subsubsection*{Aufgaben bis zum Tutorentreffen 2014-10-29}
\begin{itemize}
\item Till: Überarbeitung der Anwendungsfälle und des Use-Case-Diagrammes
\item Philipp und Max: Überarbeitung von Top-Level-Architektur und Kontextmodell
\end{itemize}

\subsubsection*{Aufgaben bis nächste Woche}
\begin{itemize}
\item Chris: Fertigstellung des Pflichtenheftes
\item Philipp J.: Fertigstellung der Sequenzdiagramme
\item Philipp W. und Max: Fertigstellung des Analyse-Klassendiagrammes
\item Till: Hilfe beim Analyse-Klassendiagramm
\item Alle: Ansehen des Salespoint-Frameworks und der Programmierbeispiele Guestbook und Videoshop
\end{itemize}

\end{document}