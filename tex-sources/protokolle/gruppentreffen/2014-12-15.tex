\documentclass[12pt,a4paper]{article}

\usepackage[utf8]{inputenc}
\usepackage[english,german]{babel}
\usepackage{amsmath}
\usepackage{amsfonts}
\usepackage{amssymb}
\usepackage{amsthm}
\usepackage[left=2cm,right=2cm,top=2cm,bottom=2cm]{geometry}
\setlength{\parindent}{0pt}

\author{Protokollführer: Till Köhler}
\title{Sitzungsprotokoll Gruppentreffen 2014-12-15}
\date{}

\begin{document}

\maketitle

\subsection*{Anwesenheit}
\medskip
\begin{itemize}
\item Philipp Waack (Chefprogrammierer)
\item Philipp Jäschke (Assistent)
\item Christoph Kepler (Operator)
\item Maximilian Dühr (Testverantwortlicher)
\item Till Köhler (Sekretär)
\end{itemize}

\subsection*{Dauer des Treffens}
\medskip
\begin{itemize}
\item Beginn: 13:00 Uhr
\item Ende: 18:30 Uhr
\end{itemize}

\noindent\rule{\textwidth}{1pt}

\subsection*{Tagespunkte}
\medskip

\subsubsection*{Vorstellung der einzelnen Arbeitsfortschritte}
\begin{itemize}
\item Philipp W.: Fertigstellung der Artikelverwaltung für Bücher inklusive Artikelansicht, -änderung und -löschung
\item Philipp J.: Fertigstellung des Warenkorbes, Überarbeitung der Bestellungsverwaltung, des Inventars und der Statistiken
\item Max: Überarbeitung des Kalenders und der Raumverwaltung
\item Till: Fertigstellung der Registrierung; Überarbeitung der Nutzerverwaltung inklusive Profilansicht, -änderung und sperrung; Überarbeitung der Rollenverwaltung
\item Chris: Überarbeitung der GUI; Hilfe bei Nutzer-, Artikel- und Raumverwaltung
\end{itemize}

\subsubsection*{Auswertung der einzelnen Arbeitsfortschritten}
\begin{itemize}
\item Klärung des weiteren Vorhabens
\item Klärung individueller Fragen und Probleme
\end{itemize}

\subsubsection*{Gemeinsames Programmieren}
\begin{itemize}
\item individuelle Arbeit am eigenen Implementierungsbereich
\end{itemize}

\subsubsection*{Aufgaben bis zum Tutorentreffen 2014-12-10}
\begin{itemize}
\item Alle: Fertigstellung des eigenen Implementierungsbereiches
\end{itemize}

\subsubsection*{Aufgaben bis Ende der Woche}
\begin{itemize}
\item Alle: kleinere Verbesserungen am eigenen Implementierungsbereich; Anfang von JUnit-Tests, Entwurf-Klassendiagramm, JavaDoc und der Entwicklerdokumentation
\end{itemize}

\end{document}