\documentclass[12pt,a4paper]{article}

\usepackage[utf8]{inputenc}
\usepackage[english,german]{babel}
\usepackage{amsmath}
\usepackage{amsfonts}
\usepackage{amssymb}
\usepackage{amsthm}
\usepackage[left=2cm,right=2cm,top=2cm,bottom=2cm]{geometry}
\setlength{\parindent}{0pt}

\author{Protokollführer: Till Köhler}
\title{Sitzungsprotokoll Tutorentreffen 2014-12-17}
\date{}

\begin{document}

\maketitle

\subsection*{Anwesenheit}
\medskip
\begin{itemize}
\item Christoph Biering (Kunde, Tutor)
\item Philipp Waack (Chefprogrammierer)
\item Philipp Jäschke (Assistent)
\item Christoph Kepler (Operator)
\item Maximilian Dühr (Testverantwortlicher)
\item Till Köhler (Sekretär)
\end{itemize}

\subsection*{Dauer des Treffens}
\medskip
\begin{itemize}
\item Beginn: 11:10 Uhr
\item Ende:
\end{itemize}

\noindent\rule{\textwidth}{1pt}

\subsection*{Tagespunkte}
\medskip

\subsubsection*{Absprache zur Endpräsentation}
\begin{itemize}
\item möglich ab dem 16.01.
\item Christoph B. erstellt eine Doodle-Umfrage zur Terminfindung
\end{itemize}

\subsubsection*{Absprache zur Implementierungsarbeit}
\begin{itemize}
\item bis Freitag, den 19.12.: endgültige Fertigstellung (Verpackung von WAR-Datei und Pflichtenheft als ZIP-Ordner und Upload bei einem Freehoster)
\item bis Sonntag, den 21.12.: endgültige Fertigstellung von JavaDoc und JUnit-Tests
\end{itemize}

\subsubsection*{Vorstellung der Implementierung}
\begin{itemize}
\item Überprüfung der Muss-, Kann- und Wunschkritrien anhand des Pflichtenheftes
\item Änderungen nötig:
\begin{itemize}
\item Fehlermeldungen, bei nicht getätigten Änderungen
\item beim Ändern der Rollen: Anzeige nur der für den Nutzer hinzufügbaren oder entfernbaren Rollen
\item Design der Profilseiten
\item Überarbeitung der Erstellung von CD's und DVD's (Dropdown zur Auswahl der Kategorien)
\item Fehlerbehebung bei Änderung eines Artikels
\item Anzeige nur der zukünftige Lesungen für Kunden bzw. Gäste im Kalender
\item Fehlerbehebung im Kalender (Weiterleitung und Hinzufügen von Lesungen mit gleicher Zeit und gleichem Raum)
\item Überarbeitung des Kalender-Designs (Dropdown beim Hinzufügen von Lesungen)
\item Verbesserungen der Suche (contains-Anweisungen)
\item Ausgabe der Rechnungen als PDF-Datei
\item Überarbeitung der Statistik (Hinzufügen von Einkäufen, Hinzufügen einer Gesamtbilanz zusätzlich zur wöchentlichen Bilanz)
\item Hinzufügen einer Nachbestellung für Artikel
\item Hinzufügen der Sitzanzahl für einzelne Räume
\item Hinzufügen der Kategoriebearbeitung (Löschen und Hinzufügen einzelner Kategorien)
\end{itemize}
\item Feedback: noch Einiges zu tun
\end{itemize}

\subsection*{Aufgaben bis Ende der Woche}
\medskip
\begin{itemize}
\item bis Freitag, den 19.12.: endgültige Fertigstellung der Implementierung (Verpackung von WAR-Datei und Pflichtenheft als ZIP-Ordner und Upload bei einem Freehoster)
\item bis Sonntag, den 21.12.: endgültige Fertigstellung von JavaDoc und JUnit-Tests
\end{itemize}

\subsection*{Aufgaben bis nach den Weihnachtsferien}
\medskip
\begin{itemize}
\item endgültige Fertigstellung des GUI-Designs
\item Behebung eventuell auftretender Fehler im Quellcode
\item Implementierung des Kunden-Extrawunsches
\item Aktualisierung des Entwurfs-Klassendiagrammes
\item Anfang von Cross-Testing, Entwickler- und Anwenderdokumentation
\end{itemize}

\end{document}