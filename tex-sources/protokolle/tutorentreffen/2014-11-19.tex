\documentclass[12pt,a4paper]{article}

\usepackage[utf8]{inputenc}
\usepackage[english,german]{babel}
\usepackage{amsmath}
\usepackage{amsfonts}
\usepackage{amssymb}
\usepackage{amsthm}
\usepackage[left=2cm,right=2cm,top=2cm,bottom=2cm]{geometry}
\setlength{\parindent}{0pt}

\author{Protokollführer: Till Köhler}
\title{Sitzungsprotokoll Tutorentreffen 2014-11-19}
\date{}

\begin{document}

\maketitle

\subsection*{Anwesenheit}
\medskip
\begin{itemize}
\item Christoph Biering (Kunde, Tutor)
\item Philipp Waack (Chefprogrammierer)
\item Philipp Jäschke (Assistent)
\item Christoph Kepler (Operator)
\item Maximilian Dühr (Testverantwortlicher)
\item Till Köhler (Sekretär)
\end{itemize}

\subsection*{Dauer des Treffens}
\medskip
\begin{itemize}
\item Beginn: 11:10 Uhr
\item Ende: 12:40 Uhr
\end{itemize}

\noindent\rule{\textwidth}{1pt}

\subsection*{Tagespunkte}
\medskip

\subsubsection*{Auswertung des GUI-Prototypen}
(von Chris erstellt)
\begin{itemize}
\item GUI-Prototyp wegen Speicherfehler bisher noch unvollendet (Erklärung des zukünftigen Vorhabens durch Chris)
\item Feedback: bisher super
\item Website muss sofort als Website einer Buchhandlung erkennbar sein
\end{itemize}

\subsubsection*{Auswertung des Entwurfs-Klassendiagramm}
(von Philipp W. und Max erstellt)
\begin{itemize}
\item Feedback: schon gut, aber noch einige Änderungen nötig
\item Aufsplitten der UML-Diagramme in verschiedene kleinere MagicDraw-Projekte (um Nummerierung der Klassen zu verhindern)
\item Umbenennung aller Managements in ''...Controller''
\item Konsistenz aller Klassen, Methoden und Attribute in englischer Sprache
\item korrekte Umsetzung externer Klassen mit MagigDraw: externe Klassen grün, Interfaces wie normale Klassen  (aber mit ''interface''-Bezeichnung)
\item sinnvolle Umbenennung der Enumeration ''ArticleID''
\item Übersichtlichkeit: (erneutes) Auseinanderziehen der Klassen, Pfeile nicht über andere Klassen zeichnen, Controllerschicht auf extra Ebene mit Abgrenzung zur Modellschicht
\item Rollenklassen entfernen, stattdessen Umbenennung von ''Guest in ''User'' Verwendung der Salespoint-Klasse ''UserAccount'' (evtl. Enumeration für die Rollen?)
\end{itemize}

\subsubsection*{Auswertung der Akzeptanzfälle im Pflichtenheft}
(von Till erstellt)
\begin{itemize}
\item Feedback: super
\end{itemize}

\subsubsection*{Neue Aufgabenverteilung zu den Anwendungsprototypen}
\begin{itemize}
\item mögliche Module: LogIn/Registrierung, Nutzerverwaltung, Artikelverwaltung, Warenkorbverwaltung, Kalenderverwaltung und (Raumverwaltung)
\item Chris: LogIn/Registrierung, Einbinden der Module in den GUI-Prototypen
\item Till: Nutzerverwaltung
\item Philipp W.: Artikelverwaltung
\item Philipp J.: Warenkorbverwaltung
\item Max: Kalenderverwaltung
\end{itemize}

\subsubsection*{Informationen zur Zwischenpräsentation}
\begin{itemize}
\item Termin: 26.11,2014 11:10 Uhr - 12:40 Uhr
\item Zeitdauer von insgesamt 30 Minuten (15 Minuten Vortrag zu bisherigem Projektverlauf mit Übergang von Analyse zu Entwurf unter Verwendung verschiedener UML-Diagramme, 15 Minuten Vorstellung von GUI und Anwendungsprototypen)
\item Vortrag mithilfe einer Beamer-Präsentation
\end{itemize}

\subsection*{Aufgaben bis zur nächsten Woche}
\medskip
\begin{itemize}
\item Philipp W. und Max: Überarbeitung des Entwurfs-Klassendiagrammes (spätestens bis Montag zur nochmaligen Korrektur des Tutors)
\item Chris: Fertigstellung des GUI-Prototypen (spätestens bis Montag zur nochmaligen Korrektur des Tutors)
\item Alle: Erstellung der Anwendungsprototypen mit Mindestfunktionalitäten; Ausarbeitung der Zwischenpräsentation
\end{itemize}

\end{document}