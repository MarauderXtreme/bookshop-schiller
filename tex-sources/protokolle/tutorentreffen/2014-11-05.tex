\documentclass[12pt,a4paper]{article}

\usepackage[utf8]{inputenc}
\usepackage[english,german]{babel}
\usepackage{amsmath}
\usepackage{amsfonts}
\usepackage{amssymb}
\usepackage{amsthm}
\usepackage[left=2cm,right=2cm,top=2cm,bottom=2cm]{geometry}
\setlength{\parindent}{0pt}

\author{Protokollführer: Till Köhler}
\title{Sitzungsprotokoll Tutorentreffen 2014-11-05}
\date{}

\begin{document}

\maketitle

\subsection*{Anwesenheit}
\medskip
\begin{itemize}
\item Christoph Biering (Kunde, Tutor)
\item Philipp Waack (Chefprogrammierer)
\item Philipp Jäschke (Assistent) - ca. 5 Minuten später
\item Christoph Kepler (Operator)
\item Maximilian Dühr (Testverantwortlicher) - ca. 5 Minuten später
\item Till Köhler (Sekretär)
\end{itemize}

\subsection*{Dauer des Treffens}
\medskip
\begin{itemize}
\item Beginn: 11:00 Uhr
\item Ende: 12:00 Uhr
\end{itemize}

\noindent\rule{\textwidth}{1pt}

\subsection*{Tagespunkte}
\medskip

\subsubsection*{Änderung der Zeit für die Tutorentreffen}
\begin{itemize}
\item ab sofort erst ab 11:10 Uhr
\item Termin weiterhin wöchentlich am Mittwoch
\end{itemize}

\subsubsection*{Auswertung der Sequenzdiagramme}
(von Philipp J. erstellt)
\begin{itemize}
\item Feedback: sehr gut
\item gute Konsistenz zu den bisher erstellten Diagrammen
\end{itemize}

\subsubsection*{Auswertung des Pflichtenheftes}
(von Chris erstellt)
\begin{itemize}
\item Feedback: Struktur gut, aber es fehlt noch einiges
\item Fachliche Übersicht: Ergänzung der genauen Umsetzung der Aufgabenstellung in Fachsprache
\item UML-Diagramme: keine genauen Beschreibungen zu den Diagrammen nötig
\item Anwendungsfälle: Beschreibung der Akteure in ausformulierten Sätzen; Anwendungsfallbeschreibungen ausführlicher und in Tabellenform (mit Identifiern)
\item Anforderungen: Aufsplitten und jeden Punkt einzeln spezifizieren; Bezeichner für die einzelnen Kriterien einführen
\item Dialoge: Ergänzen des Storyboards; Aufsplitten zur genaueren Ansicht
\item Datenmodell: Ergänzung der Beschreibungen zu den einzelnen Klassen
\item Akzeptanzfälle (= Dinge, die später einen Test erfordern): Ergänzung aller Akzeptanzfälle; ausführlich und in Tabellenform (mit Identifiern zugehörig zu den Anwendungsfällen)
\item Offene Punkte: nicht zwingend notwendig
\item Überarbeitung bis nächste Woche (am besten schon vorher)
\end{itemize}

\subsubsection*{Auswertung des Analyse-Klassendiagrammes}
(von Philipp W. und Max erstellt)
\begin{itemize}
\item Christoph B. hat die aktuelle Version nicht erhalten (Fehler beim ''Syncen'' im GitHub)
\item Hochladen der neuen Version so schnell wie möglich
\end{itemize}

\subsubsection*{Abwesenheit von Chris zum Tutorentreffen 2014-11-12}
\begin{itemize}
\item Chris kommt nächste Woche voraussichtlich gar nicht oder sonst erst später zum Tutorentreffen
\item Beginn trotzdem um 11:10 Uhr
\end{itemize}

\subsubsection*{Ausblick in die Entwurfsphase}
\begin{itemize}
\item Dauer der Entwurfsphase: 3 Wochen
\item in 1 Woche:
\begin{itemize}
\item Abgabe der Zustandsdiagramme (nur für bereits umsetzbare Dinge, wie LogIn oder Hinzufügen von Lesungen in den Kalendar)
\item selbsständiges Bearbeiten von je 3 Übungsaufgaben zum Programmierbeispiel Videoshop (werden von Christoph B. per E-Mail gesendet)
\end{itemize}
\item in 2 Wochen: 
\begin{itemize}
\item Abgabe eines entwurfstauglichen Klassendiagrammes mit Einarbeitung genauerer Salespoint-Klassen
\item Fertigstellung des GUI-Prototypen
\end{itemize}
\item in 3 Wochen:
\begin{itemize}
\item entgültige Fertigstellung des Entwurfs-Klassendiagrammes
\item Fertigstellung kleinerer Anwendungsprototypen
\end{itemize}
\item Zwischenpräsentation (evtl. bereits vor Ende der Entwurfsphase): Ausarbeiten einer Präsentation; GUI-Prototyp und kleinere Anwendungsprototypen müssen bereits lauffähig sein (!)
\end{itemize}

\subsection*{Aufgaben bis zur nächsten Woche}
(Abgabe kommenden Dienstag, 23:00 Uhr)
\medskip
\begin{itemize}
\item Chris: Überarbeitung des Pflichtenheftes
\item Philipp W. und Max: Hochladen des Analyse-Klassendiagrammes; Beginn des Entwurfs-Klassendiagrammes
\item Philipp J. und Till: Fertigstellung der Zustandsdiagramme
\item Alle: Bearbeitung von je 3 Aufgaben für das Programmierbeispiel Videoshop
\end{itemize}

\end{document}
