\documentclass[12pt,a4paper]{article}

\usepackage[utf8]{inputenc}
\usepackage[english,german]{babel}
\usepackage{amsmath}
\usepackage{amsfonts}
\usepackage{amssymb}
\usepackage{amsthm}
\usepackage[left=2cm,right=2cm,top=2cm,bottom=2cm]{geometry}
\setlength{\parindent}{0pt}

\author{Protokollführer: Till Köhler}
\title{Sitzungsprotokoll Tutorentreffen 2014-10-29}
\date{}

\begin{document}

\maketitle

\subsection*{Anwesenheit}
\medskip
\begin{itemize}
\item Christoph Biering (Kunde, Tutor)
\item Philipp Waack (Chefprogrammierer)
\item Philipp Jäschke (Assistent)
\item Christoph Kepler (Operator)
\item Maximilian Dühr (Testverantwortlicher)
\item Till Köhler (Sekretär)
\end{itemize}

\subsection*{Dauer des Treffens}
\medskip
\begin{itemize}
\item Beginn: 11:00 Uhr
\item Ende: 12:15 Uhr
\end{itemize}

\noindent\rule{\textwidth}{1pt}

\subsection*{Tagespunkte}
\medskip

\subsubsection*{Arbeitszeiten}
\begin{itemize}
\item sind von der Website verlinkt
\item müssen im MediaWiki des public-Repositories aktuell gehalten werden (!)
\item Philipp W., Max und Philipp J.: Arbeitszeiten eintragen
\end{itemize}

\subsubsection*{Auswertung des Use-Case-Diagrammes}
(von Till erstellt)
\begin{itemize}
\item Feedback: sehr gut
\item Kleinigkeiten, die noch verändert werden sollten:
\begin{itemize}
\item Customer soll sich nicht nochmal registrieren dürfen (als Kommentar an den Vererbungspfeil schreiben)
\item Einführung eines Chefs
\item Administrator und UserManager muss Employee neue Rollen zuweisen dürfen
\item für CD's und DVD's extra Kategorien einführen
\item Übersicht über bestellte Artikel muss für Customer abrufbar sein
\item Einführung eines Warenbestands
\end{itemize}
\end{itemize}

\subsubsection*{Auswertung der Top-Level-Architektur und des Kontextmodells}
(von Philipp W. und Max erstellt)
\begin{itemize}
\item Feedback: sehr gut
\end{itemize}

\subsubsection*{Absprache zu den Sequenzdiagrammen}
(wird von Philipp erstellt)
\begin{itemize}
\item Darstellung von wichtigen Anwendungsbeispielen
\item müssen nur über die größeren Module angefertigt werden
\item keine nähere Spezifikation der Akteure (aber dargestellte Akteure dazuschreiben)
\end{itemize}

\subsubsection*{Absprache zum angefangenen Analyse-Klassendiagramm}
(von Philipp W. und Max erstellt)
\begin{itemize}
\item Feedback: guter Ansatz
\item grobe Struktur ist wie im bereits fertiggestellten Use-Case-Diagrammen eingehalten, muss aber noch an neueste Versionen angepasst werden
\item einige Änderungen wegen bereits bereitgestellten Salespoint-Klassen
\item zusätzlich mit darzustellen: Zusammenhang von UserManagement, Warenkorb und den Salespoint-Klassen
\item Einbauen von CD- und DVD-Verwaltung mit extra Kategorien
\end{itemize}

\subsubsection*{Absprache zum Pflichtenheft}
\begin{itemize}
\item Muss-Kriterien:
\begin{itemize}
\item Akteure: Mitarbeiter, nicht eingeloggter Nutzer, eingeloggter Nutzer, Chef, Administrator, Personalverwalter
\item Nutzer kann mehrere Rollen haben
\item es muss immer mindestens ein Administrator im System existieren
\item Personalverwalter kann Rollen der anderen Nutzer verändern, außer die des Admins
\item nicht eingeloggter Nutzer kann Artikel suchen und in den Warenkorb legen, aber nicht kaufen
\item kaufen können nur registrierte Kunden
\item Kunde kann sich Übersicht über bereits gekaufte Artikel anzeigen lassen
\item zu jeder Bestellung wird eine pdf-Rechnung erzeugt und für den Kunden online abrufbar gemacht
\item Administrator kann Kategorien und Räume anlegen, löschen und bearbeiten
\item Mitarbeiter kann Artikelbestand sehen und gegebenenfalls nachbestellen
\item Verfügbarkeit von wöchentlichen Verkaufsbilanzen für Administrator und Chef
\item extra Kategorien für CD's und DVD's
\item Existenz eines Kalenders mit wöchentlichen Lesungen
\item wenn ein Lesungsraum bzw. eine Lesungszeit bereits vergeben ist muss der verwaltende Angestellte eine Meldung erhalten
\end{itemize}
\item Kann-Kriterien:
\begin{itemize}
\item Meldung, wenn Benutzername bereits vergeben ist
\item Mldung, wenn E-Mail syntaktisch nicht richtig ist (z.B. fehlendes "@")
\item pdf-Rechnung per E-Mail an Kunden schicken
\item Artikel haben verschiedene Kaufs- und Verkaufspreise
\end{itemize}
\item Wunsch-Kriterien:
\begin{itemize}
\item Möglichkeit Artikel nach Kategorien zu filtern
\item grafische Visualisierung der Bilanzen
\end{itemize}
\end{itemize}

\subsection*{Aufgaben bis zur nächsten Woche}
(Abgabe kommenden Dienstag, 23:00 Uhr)
\medskip
\begin{itemize}
\item Chris: Fertigstellung Pflichtenheft
\item Philipp W. und Max: Fertigstellung Analyse-Klassendiagramm; Fertigstellung Storyboard
\item Philipp J.: Fertigstellung Sequenzdiagramm
\item Till: Kleinigkeiten am Use-Case-Diagramm und der Anwendungsfallübersicht ausbessern
\item Alle: selbstständiges Beschäftigen mit dem Guestbook
\end{itemize}

\end{document}