\documentclass[12pt,a4paper]{article}

\usepackage[utf8]{inputenc}
\usepackage[english,german]{babel}
\usepackage{amsmath}
\usepackage{amsfonts}
\usepackage{amssymb}
\usepackage{amsthm}
\usepackage[left=2cm,right=2cm,top=2cm,bottom=2cm]{geometry}
\setlength{\parindent}{0pt}

\author{Protokollführer: Till Köhler}
\title{Sitzungsprotokoll Tutorentreffen 2014-12-10}
\date{}

\begin{document}

\maketitle

\subsection*{Anwesenheit}
\medskip
\begin{itemize}
\item Christoph Biering (Kunde, Tutor)
\item Philipp Waack (Chefprogrammierer)
\item Philipp Jäschke (Assistent)
\item Christoph Kepler (Operator)
\item Maximilian Dühr (Testverantwortlicher)
\item Till Köhler (Sekretär)
\end{itemize}

\subsection*{Dauer des Treffens}
\medskip
\begin{itemize}
\item Beginn: 11:05 Uhr
\item Ende: 12:20 Uhr
\end{itemize}

\noindent\rule{\textwidth}{1pt}

\subsection*{Tagespunkte}
\medskip

\subsubsection*{Absprache zu den JUnit-Tests}
\begin{itemize}
\item jeder schreibt die Tests für seine eigenen Implementierungen
\item der Test-Verantwortliche ist dafür zuständig, dass die Tests in Kombination zueinander ausführbar sind
\item alle Akzeptanzfälle müssen mit den JUnit-Tests im Wesentlichen abgedeckt sein
\end{itemize}

\subsubsection*{Individuelle Auswertung der bisherigen Programmierfortschritte}
\begin{itemize}
\item Vorstellung des eigenen Programmierfortschritts
\item Ausblick auf weiteres Vorhaben
\item Klärung von Fragen und Problemen zur Implementierung
\end{itemize}

\subsection*{Aufgaben bis zur nächsten Woche}
\medskip
\begin{itemize}
\item Alle: individuelle Arbeit am eigenen Implementierungsbereich (Aufteilung wie zum Tutorentreffen 2014-11-12 besprochen); Fertigstellung der Implementierung
\item bis Sonntag, den 14.12.2014: Erzeugung einer WAR-Datei und Senden an Christoph B.
\item bis Sonntag, den 21.12.2014: Fertigstellung JavaDoc, JUnit-Tests und Anfang der Entwicklerdokumentation
\end{itemize}

\end{document}