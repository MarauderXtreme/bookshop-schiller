\documentclass[12pt,a4paper]{article}

\usepackage[utf8]{inputenc}
\usepackage[english,german]{babel}
\usepackage{amsmath}
\usepackage{amsfonts}
\usepackage{amssymb}
\usepackage{amsthm}
\usepackage[left=2cm,right=2cm,top=2cm,bottom=2cm]{geometry}
\setlength{\parindent}{0pt}

\author{Protokollführer: Till Köhler}
\title{Sitzungsprotokoll Tutorentreffen 2014-11-12}
\date{}

\begin{document}

\maketitle

\subsection*{Anwesenheit}
\medskip
\begin{itemize}
\item Christoph Biering (Kunde, Tutor)
\item Philipp Waack (Chefprogrammierer)
\item Philipp Jäschke (Assistent)
\item Maximilian Dühr (Testverantwortlicher)
\item Till Köhler (Sekretär)
\end{itemize}

\subsection*{Dauer des Treffens}
\medskip
\begin{itemize}
\item Beginn: 11:15 Uhr
\item Ende: 12:15 Uhr
\end{itemize}

\noindent\rule{\textwidth}{1pt}

\subsection*{Tagespunkte}
\medskip

\subsubsection*{Auswertung der Zustandsdiagramme}
(von Philipp J. und Till erstellt)
\begin{itemize}
\item Feedback: super
\item letzte Version von Login/Registration wurde erst nach 23 Uhr hochgeladen (nur eine winzige Änderung zu vorher)
\end{itemize}

\subsubsection*{Auswertung des Analyse-Klassendiagrammes}
(von Philipp W. und Max erstellt)
\begin{itemize}
\item Feedback: super
\item ''Auseinanderziehen'' der Klassen zur besseren Übersichtlichkeit
\end{itemize}

\subsubsection*{Auswertung des groben Entwurfs-Klassendiagrammes}
(von Philipp W. und Max erstellt)
\begin{itemize}
\item Feedback: schon gut
\item Weglassen von LogIn und LogOut (wird vom Spring-Framework übernommen)
\item eventuelles Ergänzen der Controller (Christoph B. fragt nach)
\item ''Auseinanderziehen'' der Klassen zur besseren Übersichtlichkeit
\item Entwurf-Klassendiagramm soll im Gegensatz zum Analyse-Klassendiagramm permanent angepasst werden
\end{itemize}

\subsubsection*{Auswertung des Pflichtenheftes}
(von Chris erstellt)
\begin{itemize}
\item Feedback: sehr gut, nur Überarbeitung der Akzeptanzfälle noch notwendig
\item Akzeptanzfälle = die Punkte, die selbst (mit JUnit) getestet werden müssen und die dem Kunden bei der Abnahme gezeigt werden
\item umfangreicher und Ergänzen aller Positiv-Tests
\end{itemize}

\subsubsection*{Auswertung der Aufgaben zum Videoshop}
\begin{itemize}
\item Klärung individueller Fragen und Probleme
\item Absprache zu den Problemen in Aufgabe 3
\item außer Aufgabe 3, alles funktionstüchtig
\end{itemize}

\subsubsection*{Vorläufige Aufteilung zu den Anwendungsprototypen}
\begin{itemize}
\item Philipp W.: ArticleManagement
\item Philipp J.: SaleManagement
\item Max: ReadingManagement
\item Till: UserManagement
\item Chris kümmert sich vorerst, um den GUI-Prototypen
\end{itemize}

\subsection*{Aufgaben bis zur nächsten Woche}
(Abgabe kommenden Dienstag, 23:00 Uhr)
\medskip
\begin{itemize}
\item Chris: Fertigstellung des GUI-Prototypen inklusive fertigem Design
\item Philipp J. und Till: Vervollständigung der Akzeptanztestfälle im Pflichtenheft
\item Philipp W. und Max: Fertigstellung des Entwurfs-Klassendiagrammes
\item Alle: genaueres Auseinandersetzen mit den eigenen Anwendungsprototypen
\end{itemize}

\end{document}