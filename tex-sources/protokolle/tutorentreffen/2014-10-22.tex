\documentclass[12pt,a4paper]{article}

\usepackage[utf8]{inputenc}
\usepackage[english,german]{babel}
\usepackage{amsmath}
\usepackage{amsfonts}
\usepackage{amssymb}
\usepackage{amsthm}
\usepackage[left=2cm,right=2cm,top=2cm,bottom=2cm]{geometry}
\setlength{\parindent}{0pt}

\author{Protokollführer: Christoph Kepler, Überarbeitet von: Till Köhler}
\title{Sitzungsprotokoll Tutorentreffen 2014-10-22}
\date{}

\begin{document}

\maketitle

\subsection*{Anwesenheit}
\medskip
\begin{itemize}
\item Christoph Biering (Kunde, Tutor)
\item Philipp Waack (Chefprogrammierer)
\item Philipp Jäschke (Assistent)
\item Christoph Kepler (Operator)
\item Maximilian Dühr (Sekretär) - ca. 5 Minuten später
\item Till Köhler (Testverantwortlicher)
\end{itemize}

\subsection*{Dauer des Treffens}
\medskip
\begin{itemize}
\item Beginn: 18:30 Uhr
\item Ende: 19:30 Uhr
\end{itemize}

\noindent\rule{\textwidth}{1pt}

\subsection*{Tagespunkte}
\medskip

\subsubsection*{Klärung der Kommunikation}
\begin{itemize}
\item privat über Facebook, Skype, usw. möglich
\item offizielle Kommunikation über die FusionForge-Mailingliste
\end{itemize}

\subsubsection*{Änderung des Termins für gemeinsame Tutorentreffen}
\begin{itemize}
\item ab nächster Woche: wöchentlich, Mittwoch 11:00 - 12:30 Uhr (Treff: Foyer INF-Bau)
\end{itemize}

\subsubsection*{Stand der Einrichtung der Software}
\begin{itemize}
\item Einrichtung der Software bei jedem abgeschlossen
\item Verwendung von PHP: Definitiv nicht möglich!
\end{itemize}

\subsubsection*{Abgabe der Aufgaben}
\begin{itemize}
\item soll über das SCM in FusionForge oder GitHub laufen (Abstimmung: siehe unten)
\item Anlegung einer vernünftigen Ordnerstruktur notwenig
\end{itemize}

\subsubsection*{Hosting des Programms}
\begin{itemize}
\item nur auf dem Localhost möglich
\item Programm wird zum zeigen nur auf den jeweiligen Rechnern ausgeführt
\item auf dem Server wird nichts liegen
\end{itemize}

\subsubsection*{Auswertung des Projektplanes}
\begin{itemize}
\item Christoph B. in Absprache mit Philipp W.
\item erster Eindruck: Projektplan in Ordnung
\end{itemize}

\subsubsection*{Design des Programmes}
\begin{itemize}
\item Design obliegt uns selbst
\item Keine Vorgaben vom Kunden
\item  Lizenzrechte beachten (!)
\item keine anstößigen oder rechtsverletzenden Inhalte (!)
\end{itemize}

\subsubsection*{Klärung allgemeiner Funktionalitäten des Programmes}
\begin{itemize}
\item Warenkorb notwendig (im Salespoint bereits vorhanden)
\item Kaufen von Artikeln nur als registrierter Nutzer möglich
\item Benutzerkonten für Kunden und Angestellte
\item evtl. Bestände bzw. Bilanzen ausgeben lassen
\item Akteure:
\begin{itemize}
\item Gast
\item Kunde
\item Angestellter
\item Administrator
\item Angestellte, die nur für die Benutzerverwaltung, ... zuständig sind (?)
\end{itemize}
\item Sperrung, Entsperrung, ... des Accounts notwendig
\item Raumverwaltung für den Kalender notwendig
\end{itemize}

\subsubsection*{Programmierstil}
\begin{itemize}
\item möglichst übersichtlich und wiederverwendbar
\item Ziel: Testen des eigenen Programmierstils
\item Benutzung von Design-Patterns so oft es geht  
\end{itemize}

\subsubsection*{Pflichtenheft}
\begin{itemize}
\item übernimmt Chris
\item Christoph B. schickt Chris eine Vorlage für das Pflichtenheft
\item muss Pflicht-, Kann- und Wunschkriterien enthalten
\end{itemize}

\subsubsection*{Abstimmung zur Nutzung von GitHub}
\begin{itemize}
\item Pro: 3, Kontra: 0, Enthaltung: 2
\item Antrag wird angenommen
\item Chris kümmert sich um die Erstellung eins GitHub-Repositories
\end{itemize}

\subsection*{Aufgaben bis zur nächsten Woche}
\medskip
\begin{itemize}
\item Chris: Erstellung eines GitHub-Repositories; Einrichtung eines MediaWikis im GitHub; Verbesserung der Website und Einfügung von CSS-Code; Anfang des Pflichtenheftes
\item Till: Fertigstellung der Anwendungsfallübersicht und des Use-Case-Diagrammes
\item Philipp W. und Max: Fertigstellung des Kontextmodells und der Top-Level-Architektur; Anfang des Analyse-Klassendiagrammes; Vorbereitung CRC-Karten
\item Philipp J.: Anfang des Sequenzdiagrammes
\item Alle: Erstellung von GitHub-Accounts; selbstständige Beschäftigung mit den Programmbeispielen Guestbook und Videoshop; Gedanken machen zum Corporate Design unserer Anwendung
\end{itemize}

\end{document}